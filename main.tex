\documentclass[12pt,letterpaper]{book}

%%%%%%%%%%%%%%%%%%%%%%%%%%%%%%%%%
% PACKAGE IMPORTS
%%%%%%%%%%%%%%%%%%%%%%%%%%%%%%%%%


\usepackage[tmargin=2cm,rmargin=1in,lmargin=1in,margin=0.85in,bmargin=2cm,footskip=.2in]{geometry}
\usepackage{amsmath,amsfonts,amsthm,amssymb,mathtools}
\usepackage{xfrac}
\usepackage[makeroom]{cancel}
\usepackage{mathtools}
\usepackage{bookmark}
\usepackage{enumitem}
\usepackage{hyperref,theoremref}
\usepackage[most,many,breakable]{tcolorbox}
\usepackage{xcolor}
\usepackage{varwidth}
\usepackage{varwidth}
\usepackage{etoolbox}
\usepackage{nameref}
\usepackage{multicol,array}
\usepackage{tikz-cd}
\usepackage[ruled,vlined,linesnumbered]{algorithm2e}
\usepackage{comment} 
\usepackage{import}
\usepackage{pdfpages}
\usepackage{transparent}


%%%%%%%%%%%%%%%%%%%%%%%%%%%%%%
% SELF MADE COLORS
%%%%%%%%%%%%%%%%%%%%%%%%%%%%%%

\definecolor{myg}{RGB}{56, 140, 70}
\definecolor{myb}{RGB}{45, 111, 177}
\definecolor{myr}{RGB}{199, 68, 64}
\definecolor{mytheorembg}{HTML}{F2F2F9}
\definecolor{mytheoremfr}{HTML}{00007B}
\definecolor{mylenmabg}{HTML}{FFFAF8}
\definecolor{mylenmafr}{HTML}{983b0f}
\definecolor{mypropbg}{HTML}{f2fbfc}
\definecolor{mypropfr}{HTML}{191971}
\definecolor{myexamplebg}{HTML}{F2FBF8}
\definecolor{myexamplefr}{HTML}{88D6D1}
\definecolor{myexampleti}{HTML}{2A7F7F}
\definecolor{mydefinitbg}{HTML}{E5E5FF}
\definecolor{mydefinitfr}{HTML}{3F3FA3}
\definecolor{notesgreen}{RGB}{0,162,0}
\definecolor{myp}{RGB}{197, 92, 212}
\definecolor{mygr}{HTML}{2C3338}
\definecolor{myred}{RGB}{127,0,0}
\definecolor{myyellow}{RGB}{169,121,69}
\definecolor{myexercisebg}{HTML}{F2FBF8}
\definecolor{myexercisefg}{HTML}{88D6D1}


%%%%%%%%%%%%%%%%%%%%%%%%%%%%
% TCOLORBOX SETUPS
%%%%%%%%%%%%%%%%%%%%%%%%%%%%


\setlength{\parindent}{1cm}


%================================
% DEFINITION BOX
%================================

\newtcbtheorem[number within=chapter]{definition}{Definición}{
	breakable,
	enhanced,
	colback=white,
	colframe=myr,
	arc=0pt,
	outer arc=0pt,
	fonttitle=\bfseries\sffamily,
	colbacktitle=myr,
	attach boxed title to top left={},
	boxed title style={
			enhanced,
			skin=enhancedfirst jigsaw,
			arc=3pt,
			bottom=0pt,
			interior style={fill=myr}
		},
	#1
}{def}


%================================
% THEOREM BOX
%================================

\tcbuselibrary{theorems,skins,hooks}
\newtcbtheorem[number within=section]{theorem}{Teorema}
{%
	enhanced,
	breakable,
	colback = mytheorembg,
	frame hidden,
	boxrule = 0sp,
	borderline west = {2pt}{0pt}{mytheoremfr},
	sharp corners,
	detach title,
	before upper = \tcbtitle\par\smallskip,
	coltitle = mytheoremfr,
	fonttitle = \bfseries\sffamily,
	description font = \mdseries,
	separator sign none,
	segmentation style={solid, mytheoremfr},
}
{th}



%================================
% Exercise
%================================

\tcbuselibrary{theorems,skins,hooks}
\newtcbtheorem[number within=section]{exercise}{Ejercicio}
{%
	enhanced,
	breakable,
	colback = myexercisebg,
	frame hidden,
	boxrule = 0sp,
	borderline west = {2pt}{0pt}{myexercisefg},
	sharp corners,
	detach title,
	before upper = \tcbtitle\par\smallskip,
	coltitle = myexercisefg,
	fonttitle = \bfseries\sffamily,
	description font = \mdseries,
	separator sign none,
	segmentation style={solid, myexercisefg},
}
{exer}



%================================
% Corollery
%================================
\tcbuselibrary{theorems,skins,hooks}
\newtcbtheorem[number within=section]{corollary}{Corolario}
{%
	enhanced
	,breakable
	,colback = myp!10
	,frame hidden
	,boxrule = 0sp
	,borderline west = {2pt}{0pt}{myp!85!black}
	,sharp corners
	,detach title
	,before upper = \tcbtitle\par\smallskip
	,coltitle = myp!85!black
	,fonttitle = \bfseries\sffamily
	,description font = \mdseries
	,separator sign none
	,segmentation style={solid, myp!85!black}
}
{cor}



%================================
% Lema
%================================

\tcbuselibrary{theorems,skins,hooks}
\newtcbtheorem[number within=section]{lemma}{Lema}
{%
	enhanced,
	breakable,
	colback = mylenmabg,
	frame hidden,
	boxrule = 0sp,
	borderline west = {2pt}{0pt}{mylenmafr},
	sharp corners,
	detach title,
	before upper = \tcbtitle\par\smallskip,
	coltitle = mylenmafr,
	fonttitle = \bfseries\sffamily,
	description font = \mdseries,
	separator sign none,
	segmentation style={solid, mylenmafr},
}
{lm}


%================================
% PROPOSITION
%================================

\tcbuselibrary{theorems,skins,hooks}
\newtcbtheorem[number within=section]{proposition}{Proposición}
{%
	enhanced,
	breakable,
	colback = mypropbg,
	frame hidden,
	boxrule = 0sp,
	borderline west = {2pt}{0pt}{mypropfr},
	sharp corners,
	detach title,
	before upper = \tcbtitle\par\smallskip,
	coltitle = mypropfr,
	fonttitle = \bfseries\sffamily,
	description font = \mdseries,
	separator sign none,
	segmentation style={solid, mypropfr},
}
{prop}


%================================
% Observación
%================================

\tcbuselibrary{theorems,skins,hooks}
\newtcbtheorem[number within=section]{obs}{Observación}
{%
	enhanced
	,breakable
	,colback = myyellow!10
	,frame hidden
	,boxrule = 0sp
	,borderline west = {2pt}{0pt}{myyellow}
	,sharp corners
	,detach title
	,before upper = \tcbtitle\par\smallskip
	,coltitle = myyellow!85!black
	,fonttitle = \bfseries\sffamily
	,description font = \mdseries
	,separator sign none
	,segmentation style={solid, myyellow!85!black}
}
{ob}



%================================
% Notación
%================================

\tcbuselibrary{theorems,skins,hooks}
\newtcbtheorem[number within=section]{notation}{Notación}
{%
	enhanced
	,breakable
	,colback = myg!10
	,frame hidden
	,boxrule = 0sp
	,sharp corners
	,detach title
	,before upper = \tcbtitle\par\smallskip
	,coltitle = myg!85!black
	,fonttitle = \bfseries\sffamily
	,description font = \mdseries
	,separator sign none
	,segmentation style={solid, myg!85!black}
}
{not}




%================================
% Example BOX
%================================

\newtcbtheorem[number within=section]{example}{Ejemplo}
{%
	colback = myexamplebg
	,breakable
	,colframe = myexamplefr
	,coltitle = myexampleti
	,boxrule = 0sp
	,sharp corners
	,detach title
	,before upper=\tcbtitle\par\smallskip
	,fonttitle = \bfseries
	,description font = \mdseries
	,separator sign none
	,description delimiters parenthesis
}
{ex}


\newtheoremstyle{definitionstyle}% Estilo para solucion
  {10pt}% Espaciado por encima
  {10pt}% Espaciado por debajo
  {\normalfont}% Fuente del cuerpo
  {}% Sangría
  {\bfseries}% Fuente del encabezado
  {.}% Puntuación después del encabezado
  {5pt plus 1pt minus 1pt}% Espaciado después del encabezado
  {}% Especificación del encabezado

  \theoremstyle{definitionstyle}
  \newtheorem{solution}{Solución}[section]





%%%%%%%%%%%%%%%%%%%%%%%%%%%%%%
% SELF MADE COMMANDS
%%%%%%%%%%%%%%%%%%%%%%%%%%%%%%


% deliminators
\DeclarePairedDelimiter{\abs}{\lvert}{\rvert}
\DeclarePairedDelimiter{\norm}{\lVert}{\rVert}
\DeclarePairedDelimiter{\ceil}{\lceil}{\rceil}
\DeclarePairedDelimiter{\floor}{\lfloor}{\rfloor}
\DeclarePairedDelimiter{\round}{\lfloor}{\rceil}
\newcommand{\inner}[2]{\mathinner{\langle{#1},{#2}\rangle}}
\newcommand{\BigInner}[2]{\biggl \langle{#1},{#2} \biggr \rangle }
%\usepackage[varbb]{newpxmath}  %FUENTE BONITA PARA EL DOCUMENTO

%---------------------------------------
% BlackBoard Math Fonts :-
%---------------------------------------

%Captital Letters
\newcommand{\bbA}{\mathbb{A}}	\newcommand{\bbB}{\mathbb{B}}
\newcommand{\bbC}{\mathbb{C}}	\newcommand{\bbD}{\mathbb{D}}
\newcommand{\bbE}{\mathbb{E}}	\newcommand{\bbF}{\mathbb{F}}
\newcommand{\bbG}{\mathbb{G}}	\newcommand{\bbH}{\mathbb{H}}
\newcommand{\bbI}{\mathbb{I}}	\newcommand{\bbJ}{\mathbb{J}}
\newcommand{\bbK}{\mathbb{K}}	\newcommand{\bbL}{\mathbb{L}}
\newcommand{\bbM}{\mathbb{M}}	\newcommand{\bbN}{\mathbb{N}}
\newcommand{\bbO}{\mathbb{O}}	\newcommand{\bbP}{\mathbb{P}}
\newcommand{\bbQ}{\mathbb{Q}}	\newcommand{\bbR}{\mathbb{R}}
\newcommand{\bbS}{\mathbb{S}}	\newcommand{\bbT}{\mathbb{T}}
\newcommand{\bbU}{\mathbb{U}}	\newcommand{\bbV}{\mathbb{V}}
\newcommand{\bbW}{\mathbb{W}}	\newcommand{\bbX}{\mathbb{X}}
\newcommand{\bbY}{\mathbb{Y}}	\newcommand{\bbZ}{\mathbb{Z}}

%---------------------------------------
% MathCal Fonts :-
%---------------------------------------

%Captital Letters
\newcommand{\mcA}{\mathcal{A}}	\newcommand{\mcB}{\mathcal{B}}
\newcommand{\mcC}{\mathcal{C}}	\newcommand{\mcD}{\mathcal{D}}
\newcommand{\mcE}{\mathcal{E}}	\newcommand{\mcF}{\mathcal{F}}
\newcommand{\mcG}{\mathcal{G}}	\newcommand{\mcH}{\mathcal{H}}
\newcommand{\mcI}{\mathcal{I}}	\newcommand{\mcJ}{\mathcal{J}}
\newcommand{\mcK}{\mathcal{K}}	\newcommand{\mcL}{\mathcal{L}}
\newcommand{\mcM}{\mathcal{M}}	\newcommand{\mcN}{\mathcal{N}}
\newcommand{\mcO}{\mathcal{O}}	\newcommand{\mcP}{\mathcal{P}}
\newcommand{\mcQ}{\mathcal{Q}}	\newcommand{\mcR}{\mathcal{R}}
\newcommand{\mcS}{\mathcal{S}}	\newcommand{\mcT}{\mathcal{T}}
\newcommand{\mcU}{\mathcal{U}}	\newcommand{\mcV}{\mathcal{V}}
\newcommand{\mcW}{\mathcal{W}}	\newcommand{\mcX}{\mathcal{X}}
\newcommand{\mcY}{\mathcal{Y}}	\newcommand{\mcZ}{\mathcal{Z}}

%%Para la portada
\usepackage{./resources/portadaUnam}
\usepackage[T1]{fontenc}
\usepackage[utf8]{inputenc}
\usepackage[spanish,es-nodecimaldot,es-tabla]{babel}
\usepackage{graphicx}
\usepackage{tikz} 
\usepackage{tocloft}
\usepackage{setspace}
\setlength{\parindent}{0pt}

%\input{resources/table_of_contents.tex}

%\excludecomment{solution} 

% Ligth MODE
\definecolor{back}{HTML}{FFFFFF} 
\definecolor{text}{HTML}{000000} 
\pagecolor{back}\color{text}





\begin{document}

\maketitlepage{\large\scshape Diana Avella Alaminos}{\Huge Algebra Lineal I}{Alumno:\\ \textbf{Barrientos Sánchez José Antonio}}{}{Autora del Curso: \\ \textbf{Diana Avella Alaminos}}{}{}

\newpage
\setlength{\headheight}{18.85004pt}.
\tableofcontents

\newpage

% Unidad 1 Teoria
\chapter{Espacios vectoriales}

\section{Sección 1 (Espacios vectoriales)}
\begin{definition}{Campo}{campo} 
    Sea $K$ un conjunto no vacío con dos operaciones binarias:\\ $+:K\times K\rightarrow K$ y $\cdot:K\times K\rightarrow K$. Se dice que $K$ es un campo si se cumplen las siguientes propiedades:
    \begin{enumerate}
        \item $+$ es asociativa. \label{def-campo-asociativa-aditiva}
        \item $+$ es conmutativa. \label{def-campo-conmutativa-aditiva}
        \item Existe un elemento $0_K\in K$ tal que $a+0_K=a$ para todo $a\in K$. \label{def-campo-neutro-aditivo}
        \item Para cada $a\in K$ existe un elemento $-a\in K$ tal que $a+(-a)=0$. \label{def-campo-inverso-aditivo}
        \item $\cdot$ es asociativa. \label{def-campo-asociativa-multiplicativa}
        \item $\cdot$ es conmutativa. \label{def-campo-conmutativa-multiplicativa}
        \item Existe un elemento $1\in K$ tal que $a\cdot 1=a$ para todo $a\in K$. \label{def-campo-neutro-multiplicativo}
        \item Para cada $a\in K - \{0_K\}$ existe un elemento $a^{-1}\in K$ tal que $a\cdot a^{-1}=1$. \label{def-campo-inverso-multiplicativo}
        \item $a \cdot (b+c)=a\cdot b + a\cdot c$ para todo $a,b,c\in K$. \label{def-campo-distributiva-izquierda}
    \end{enumerate}
    Llamaremos a los elementos de K escalares.
\end{definition}


\begin{example}{}{}
\begin{itemize}
    \item $\mathbb{R}$ es un campo con las operaciones usuales.
    \item $\mathbb{Q}$ es un campo con las operaciones usuales.
    \item $\mathbb{Z}$ no es un campo pues no cumple \ref{def-campo-inverso-multiplicativo}.
    \item $\mathbb{Z}_p$ con $p$ primo es un campo.
    \item $\mathbb{Q}(\sqrt{2}) = \{x + \sqrt{2}y \mid x,y \in \mathbb{Q}\}$ es un campo.
    

\end{itemize}

\end{example}

\begin{definition}{Subcampo}{subcampo}
Sean $K$ un campo, $\tilde{K} \subseteq K$. Decimos que $\tilde{K}$ es un subcampo de $K$ si $\tilde{K}$ con las operaciones restringidas de $K$ es por sí mismo un campo.
\end{definition}

\begin{example}{}{}
    \begin{itemize}
        \item $\mathbb{Q}$ es un subcampo de $\mathbb{R}$.
        \item $\mathbb{R}$ es un subcampo de $\mathbb{C}$.
    \end{itemize}
\end{example}

\begin{definition}{Espacio Vectorial}{espacio_vectorial}
    Sea $V$ un conjunto no vacío, $K$ un campo y $+:V\times V\rightarrow V$ y $\cdot:K\times V\rightarrow V$ dos operaciones. Se dice que $V$ es un espacio vectorial sobre $K $ si se cumplen las siguientes propiedades: 
    \begin{enumerate}
        \item $(u+v)+w = u+(v+w)$ para todo $u,v,w\in V$. Asociatividad.
        \item $u+v = v+u$ para todo $u,v\in V$. Conmutatividad.
        \item Existe un elemento $0_V\in V$ tal que $v+\theta_V=v$ para todo $v\in V$. Elemento neutro.
        \item Para todo $v \in V$ existe $\hat{v} \in V$ tal que $v+\hat{v}=\hat{v}+v = \theta_V$. Inverso aditivo.
        \item $1_K \cdot v = v$ para todo $v\in V$. 
        \item $\lambda \cdot(\mu \cdot v) = (\lambda \cdot \mu) \cdot v$ para todo $\lambda,\mu \in K$ y para todo $v\in V$. Distributividad.
        \item $(\lambda + \mu) \cdot v = \lambda \cdot v + \mu \cdot v$ para todo $\lambda,\mu \in K$ y para todo $v\in V$. Distributividad.
        \item $\lambda \cdot (u+v) = \lambda \cdot u + \lambda \cdot v$ para todo $\lambda \in K$ y para todo $u,v\in V$. Distributividad.
    \end{enumerate}
    Decimos que $V,+,\cdot$ es un espacio vectorial sobre el campo $K$ o un $K$-espacio vectorial. A los elementos de $V$ los llamaremos vectores.
\end{definition}


\begin{example}{}{}
    \begin{itemize}
        \item $\mathbb{R}^n$ es un $\mathbb{R}$-espacio vectorial con las operaciones usuales.
        \item $K$ campo, $K^n$ es un $K$-espacio vectorial con las operaciones usuales.
        \begin{align*}
            K^n = \{(x_1,\dots,x_n) \mid x_i \in K\} \\
            (x_1,\dots,x_n), (y_1,\dots,y_n) \in K^n \\
            (x_1,\dots,x_n) + (y_1,\dots,y_n) = (x_1+y_1,\dots,x_n+y_n) \\
            \lambda \cdot (x_1,\dots,x_n) = (\lambda x_1,\dots,\lambda x_n)
        \end{align*}

        \item $K$ campo:
        \begin{align*}
            K^{\infty} = \{(x_1, x_2, \cdots) \mid x_i \in K \, \, \forall i \in \mathbb{N}^{+}\} \\
            (x_1, x_2, \cdots), (y_1, y_2, \cdots) \in K^{\infty} \\
            (x_1, x_2, \cdots) + (y_1, y_2, \cdots) = (x_1+y_1, x_2+y_2, \cdots) \\
            \lambda \cdot (x_1, x_2, \cdots) = (\lambda x_1, \lambda x_2, \cdots)
        \end{align*} 
        Es un $K$-espacio vectorial.

        \item $K$ campo, $\mathcal{M}_{m\times n}(K)$ con $m$ renglones y $n$ columnas con las operaciones usuales de suma y producto por escalar es un $K$-espacio vectorial.

        \item $K$ campo, $K[x]$ polinomios en $x$ con coeficientes en $K$ con las operaciones usuales es un $K$-espacio vectorial.
    \end{itemize}

\end{example}

\begin{example}{}{}
    $K$ campo:
    \begin{align*}
        V &= \{f \mid f: K \rightarrow K\} \\
    \end{align*}

\end{example}

\begin{proof}
    Sean $f,g,h \in V$. 
    P.D. $(f+g)+h = f+(g+h)$.

    Por construcción $(f+g)+h$ 
\end{proof}





\begin{proposition}{}{}
    En un espacio vectorial el neutro es único.
\end{proposition}
\begin{proposition}{}{}
    En un espacio vectorial los inversos aditivos son únicos.
\end{proposition}
\begin{proposition}{}{}[Propiedades de cancelación]
    Sean $K$ un campo y $V$ un $K$-espacio vectorial. Sean $u,v,w\in V$. Entonces:
    \begin{enumerate}
        \item Si $u+v=u+w$, entonces $v=w$.
        \item Si $u+v=w+v$, entonces $u=w$.
    \end{enumerate}
\end{proposition}
\begin{proposition}{}{}
    Sean $K$ un campo y $V$ un $K$-espacio vectorial. Entonces:
    \begin{enumerate}
        \item $0_K \cdot v = \theta_V$ para todo $v\in V$.
        \item $\lambda \cdot \theta_V = \theta_V$ para todo $\lambda \in K$.
    \end{enumerate}
\end{proposition}
\begin{proposition}{}{}
    Sea $K$ un campo y $V$ un $K$-espacio vectorial. 
    
    Para todo $v\in V, (-1_K)v$ es el inverso aditivo de $v$.
\end{proposition}
\begin{corollary}{}{}
    Sean $K$ un campo y $V$ un $K$-espacio vectorial. Entonces:
    $(-\lambda)v = - (\lambda v) = \lambda (-v)$ para todo $\lambda \in K$ y para todo $v\in V$.
\end{corollary}


\begin{notation}{}{}
    Dado $v \in V$ denotaremos por $-v$ a su inverso aditivo.
\end{notation}
\begin{notation}{}{}
    $K$ denotará siempre un campo.
\end{notation}


\newpage
\section{Sección 2 (Subespacios)}
\begin{definition}{Subespacio}{subespacio} 
    Sea $V$ un $K$-espacio vectorial y $W$ un subconjunto de $V$. Decimos que $W$ es un subespacio de $V$ si:
    \begin{enumerate}[label=\alph*)]
        \item $\theta_V \in W$.
        \item Si $u,v\in W$, entonces $u+v\in W$.
        \item Si $\lambda \in K$ y $v\in W$, entonces $\lambda v \in W$.
    \end{enumerate}
\end{definition}
\begin{notation}{}{}
    $W \leq V$ denotará que $W$ es un subespacio de $V$.
\end{notation}
\begin{proposition}{}{}
    Sea $V$ un $K$-espacio vectorial y $W$ un subconjunto de $V$. $W$ es un subespacio de $V$ si y solo si $W$ con las operaciones restringidas de $V$ es un $K$-espacio vectorial.
\end{proposition}
\begin{obs}{}{}
    Si $V$ es un $K$-espacio vectorial, $W$ subconjunto de $V$, $W \leq V$ si y solo si se cumple:
    \begin{enumerate}
        \item $W \neq \varnothing$.
        \item $\lambda u + v \in W$ para todo $u,v \in W$ y para todo $\lambda \in K$.
    \end{enumerate}
\end{obs}
\begin{proposition}
    La intersección de una familia no vacía de subespacioes es un subespacio.
\end{proposition}

\begin{proof}
    Sea $V$ un $K$-espacio vectorial y $\{W_i \mid i \in I\}$ una familia no vacía de subespacios de $V$.
\end{proof}


\begin{definition}{Combinación Lineal}{}
    Sea $V$ un $K$ espacio vectorial. Considereamos $m \in \mathbb{N}^+$ y $v_1,\dots,v_m \in V$. Una \textbf{combinación lineal} de $v_1,\dots,v_m$ es una expresión de la forma:
    $$\lambda_1 v_1 + \cdots + \lambda_m v_m \quad \lambda_1, \cdots, \lambda_m \in K$$
    De modo más general, si $S$ es un subconjunto de $V$, una \textbf{combinación lineal de vectores de $S$} es un vector de la forma:
    $$\lambda_1 v_1 + \cdots + \lambda_m v_m \quad \lambda_1, \cdots, \lambda_m \in K, v_1,\dots,v_m \in S, m \in \mathbb{N}^+$$
\end{definition}
\begin{obs}{}{}
    Aunque el conjunto $S$ sea infinito, una combinación lineal de vectores de $S$ es una suma \textbf{finita} de vectores de $S$.
\end{obs}
\begin{proposition}
    Sea $V$ un $K$-espacio vectorial, $S \neq \varnothing$ un subconjunto de $V$. El conjunto de todas las combinaciones lineales de vectores de $S$ cumple lo siguiente:
    \begin{enumerate}[label=\alph*)]
        \item Es un subespacio de $V$.
        \item Contiene a $S$.
        \item Está contenido en cualquier subespacio de $V$ que contenga a $S$.
    \end{enumerate}
\end{proposition}



\newpage
\section{Sección 3}
\begin{definition}{}{}
    Sea $V$ un $K$ espacio vectorial, $S$ un subconjunto de $V$. \textbf{El subespacio de $V$ generado por $S$} es el conjunto de combinaciones lineales de $S$, si $S \neq \varnothing$, o $\{\theta_V\}$ si $S = \varnothing$. Lo denotaremos por $\langle S \rangle$ ($span(S)$ en algunos libros).
    Decimos que $S$ \textbf{genera} a $\langle V \rangle$ o que $S$ es un \textbf{conjunto generador} de $\langle V \rangle$ si $\langle S \rangle = V$.
\end{definition}
\begin{notation}{}{}
   Sean $v_1, \dots, v_n \in V$, a $\langle \{v_1, \dots, v_n\} \rangle$ se le denota por: 
    
    $\langle v_1, \dots, v_n \rangle $
\end{notation}
\begin{obs}{}{}
    Si $W \subseteq \langle S \rangle$, pero $W \neq \langle S \rangle$, entonces $S$ no genera a $W$.
\end{obs}
\begin{definition}{}{}
    Sea $V$ un $K$-espacio vectorial. Una lista $v_1, \cdots, v_m$ de vectores en $V$ es una \textbf{lista linealmente dependiente} si existen escalares $\lambda_1, \cdots, \lambda_m \in K$, no todos nulos, tales que:
    $$\lambda_1 v_1 + \cdots + \lambda_m v_m = \theta_V$$
    Decimos que es una \textbf{lista linealmente independiente} si no es linealmente dependiente, es decir si:
    $$\lambda_1 v_1 + \cdots + \lambda_m v_m = \theta_V \Rightarrow \lambda_1 = \cdots = \lambda_m = 0_K$$
    Nota: Abreviaremos \textbf{lista linealmente independiente} por \textbf{lista l.i.} y \textbf{lista linealmente dependiente} por \textbf{lista l.d.}
\end{definition}
\begin{definition}{}{}
    Sea $V$ un $K$-espacio vectorial. Un subconjunto de $S$ de $V$ es un \textbf{conjunto linealmente dependiente} si podemos encontrar $m \in \mathbb{N}^+$ y $v_1, \cdots, v_m \in S$ \textbf{distintos} y $\lambda_1, \cdots, \lambda_m \in K$, no todos nulos, tales que:
    $$\lambda_1 v_1 + \cdots + \lambda_m v_m = \theta_V$$

    Decimos que $S$ es un \textbf{conjunto linealmente independiente} si no es linealmente dependiente, es decir, si para cualquier $m \in \mathbb{N}^+$  y cualesquiera $v_1, \cdots, v_m \in S$ \textbf{distintos} y $\lambda_1, \cdots, \lambda_m \in K$, se tiene que:
    $$\lambda_1 v_1 + \cdots + \lambda_m v_m = \theta_V \Rightarrow \lambda_1 = \cdots = \lambda_m = 0_K$$
\end{definition}
\begin{obs}{}{}
    Si $S$ es un conjunto finito con $m$ vectores distintos, digamos $S =\{v_1, \cdots, v_m\}$, para ver si $S$ es l.d. o l.i. debemos ver si existen escalares $\lambda_1, \cdots, \lambda_m \in K$, no todos nulos, tales que:
    $$\lambda_1 v_1 + \cdots + \lambda_m v_m = \theta_V$$
    ó si la única forma en que se cumple lo anterior es que $\lambda_1 = \cdots = \lambda_m = 0_K$.
\end{obs}  
\begin{lemma}{Dependencia lineal}
    Sea $V$ un $K$ espacio vectorial, $v_1, \cdots, v_m$ una lista de vectores en $V$. Si $v_1, \cdots, v_m$ es una lista l.d. y $v_1 \neq \theta_v$, existe $j \in \{2, \cdots, m\}$ tal que:
    \begin{enumerate} [label=\alph*)]
        \item $v_j \in \langle v_1, \cdots, v_{j-1} \rangle$
        \item $\langle v_1, \cdots, v_m \rangle = \langle v_1, \cdots, v_{j-1}, v_{j+1}, \cdots, v_m \rangle$
    \end{enumerate}
\end{lemma}
\begin{notation}{}{}
    $\langle v_1, \cdots, v_{j-1}, v_{j+1}, \cdots, v_m \rangle$ se denota por $\langle v_1, \cdots, \hat{v_j}, \cdots, v_m \rangle$
\end{notation}
\begin{theorem}{}{}
    Sea $V$ un $K$-espacio vectorial. Si $v_1, \cdots, v_m$ es una lista de vectores en $V$ l.i., entonces todo conjunto generador de $V$ tiene al menos $m$ elementos.
\end{theorem}
\begin{corollary}{}{}
    Sea $V$ un $K$-espacio vectorial. Si existe $S$ un subconjunto finito de $V$ generador con $l$ elementos, entonces todo conjunto linealmente independiente tiene a lo más $l$ elementos. En consecuencia no existen conjuntos linealmente independientes infinitos en $V$.
\end{corollary}

\newpage
\section{Sección 4}

\begin{definition}{}{}
    Sea $V$ un $K$-espacio vectorial, $B$ subconjunto de $V$. Decimos que $B$ es una \textbf{base} de $V$ si genera a $V$ y es linealmente independiente. Decimos que $V$ es de \textbf{dimensión finita} si tiene una base finita.
\end{definition}
\begin{proposition}{}{}
    Sea $V$ un $K$-espacio vectorial. Si $V$ tiene un conjunto generador finito, entonces $V$ tiene una base finita.
\end{proposition}
\begin{corollary}{}{}
    Sea $V$ un $K$-espacio vectorial. $V$ tiene un conjunto generador finito si y solo si $V$ es de dimensión finita.
\end{corollary}
\begin{obs}{}{}
    Si $V$ es de dimensión finita, todo conjunto l.i. es finito.
\end{obs}
\begin{theorem}{}{}
    Sea $V$ un $K$-espacio vectorial de dimensión finita. Todas las bases de $V$ son finitas y tienen el mismo número de elementos.
\end{theorem}
\begin{definition}{}{}
    Si $V$ es un $K$-espacio vectorial de dimensión finita, la \textbf{dimensión} de $V$ es el número de elementos de una base de $V$. 
\end{definition}
\begin{notation}{}{}
    Si $V$ es un $K$-espacio vectorial de dimensión finita, denotaremos por $\dim_K(V)$ a la dimensión de $V$.
\end{notation}

\begin{theorem}
    Sea $V$ un $K$-espacio vectorial de dimensión finita. 
    \begin{enumerate}[label=\alph*)]
        \item Todo conjunto generador finito se puede reducir a una base.
        \item Todo conjunto l.i. se puede completar a una base.
    \end{enumerate}
\end{theorem}
\begin{corollary}{}{}
    Sea $V$ un $K$-espacio vectorial de dimensión finita, $dim_K(V) = n$. Entonces:
    \begin{enumerate}[label=\alph*)]
        \item Cualquier conjunto generador con $n$ elementos es una base de $V$.
        \item Cualquier conjunto l.i. con $n$ elementos es una base de $V$.
    \end{enumerate}
\end{corollary}
\begin{theorem}
    Sea $V$ un $K$- espacio vectorial de dimensión finita. Sea $W$ un subespacio de $V$.
    \begin{enumerate}
        \item $W$ es de dimensión finita.
        \item Toda base de $W$ se puede completar a una base de $V$.
        \item $dim_K(W) \leq dim_K(V)$.
        \item Si $dim_K(W) = dim_K(V)$, entonces $W = V$.
    \end{enumerate}
\end{theorem}

\begin{definition}{}{}
    Sea $V$ un $K$-espacio vectorial. Sean $U, W$ subespacios de $V$. La suma de $U$ y $W$ es el conjunto:
    $$U+W = \{u+w \mid u \in U, w \in W\}$$
    Generalizando, si $U_1, \cdots, U_m \leq V$, la suma de $U_1, \cdots, U_m$ es el conjunto:
    $$U_1 + \cdots + U_m = \{u_1 + \cdots + u_m \mid u_i U_i \forall i \}$$
\end{definition}
\begin{obs}{}{}
    \begin{enumerate}
        \item $U+W \leq V$.
        \item $U \subseteq U+W$ y $W \subseteq U+W$.
        \item Si $U'$ es otro subespacio que contiene a $U$ y a $W$, entonces contiene a $U+W$.
    \end{enumerate}
\end{obs}
\begin{theorem}
    Sea $V$ un $K$-espacio vectorial de dimensión finita. Sean $U,W$ subespacios de $V$. Entonces:
    $$dim_K(U+W) = dim_K(U) + dim_K(W) - dim_K(U \cap W)$$
\end{theorem}

\begin{definition}{}{}
    Sea $V$ un $K$-espacio vectorial. $U, W$ subespacios de $V$. Decimos que $U+W$ es una \textbf{suma directa} si cada $v \in U+W$ se puede escribir de manera única como $v = u+w$ con $u \in U, w \in W$.
    En general si $U_1, \cdots, U_m$ son subespacios de $V$, $U_1 + U_2 + \cdots + U_m$ es una suma directa si cada $v \in U_1 + U_2 + \cdots + U_m$ se puede escribir de manera única como $v = u_1 + u_2 + \cdots + u_m$ con $u_i \in U_i$ para todo $i$.
\end{definition}
\begin{notation}{}{}
    La suma directa de subespacios se denota por:
    $$U \oplus W, U_1 \oplus \cdots \oplus U_m $$
\end{notation}
\begin{proposition}{}{}
    Sea $V$ un $K$-espacio vectorial. $U, W$ subespacios de $V$.
    
    $U+W$ es una suma directa si y solo si $U \cap W = \{\theta_V\}$.
\end{proposition}
\newpage


% Unidad 2 Teoria
\chapter{Transformaciones lineales}
\section{Seccion 1}
\begin{definition}{}{}
    Sean $V$ y $W$ $k$-espaacios vectoriales. Una función $T: V\rightarrow W$ es una \textbf{transformación lineal de $V$ en $W$} si:
    \begin{enumerate}
        \item $T(u+v)=T(u)+T(v)$ para todo $u,v\in V$.
        \item $T(\lambda v)=\lambda T(v)$ para todo $v\in V$ y $\lambda\in k$.
    \end{enumerate}
\end{definition}
\begin{obs}{}{}
    Si $T$ es lineal, ent. $T(\theta_V) = \theta_W$.
\end{obs}
\begin{proposition}{}{}
    Sean $V,W$ $K$-espacios vectoriales, $T:V \rightarrow W$. $T$ es lineal si y solo si $T(\lambda u + v) = \lambda T(u) + T(v)$ para todo $u,v\in V$ y $\lambda\in K$.
\end{proposition}
\begin{notation}{}{}
    $V,W$ $K$-espacios vectoriales. Denotamos por $\mathcal{L}(V,W)$ al conjunto de todas las transformaciones lineales de $V$ en $W$.
    $$\mathcal{L}(V,W) = \{T: V \rightarrow W \mid T \text{ es lineal}\}$$
\end{notation}
\begin{definition}{}{}
    Sean $V, W$ $K$-espacios vectoriales, $T \in \mathcal{L}(V,W)$. Definimos el \textbf{núcleo} de $T$ como:
    $$\ker T = Nuc(T) = \{v \in V \mid T(v) = \theta_W\}$$
    La \textbf{imagen} de $T$ como:
    $$Im(T) = \{T(v) \mid v \in V\}$$
    \textbf{Nota: } Si $Nuc(T)$ es de dimensión finita, su dimensión es la \textbf{Nulidad} de $T$.
    Si $Im(T)$ es de dimensión finita, su dimensión es el \textbf{Rango} de $T$.
\end{definition}
\begin{proposition}{}{}
    Sean $V,W$ $K$-espacios vectoriales, $T \in \mathcal{L}(V,W)$. Entonces:
    \begin{enumerate}
        \item $\ker T$ es un subespacio de $V$.
        \item $Im(T)$ es un subespacio de $W$.
    \end{enumerate}
\end{proposition}

\begin{theorem}{Teorema de la dimensión}{}
        Sean $V,W$ $K$-espacios vectoriales, $T \in \mathcal{L}(V,W)$. Si $V$ es de dimensión finita, entonces $Nuc(T)$ y $Im(T)$ son de dimensión finita y:
        $$dim(V) = dim(Nuc(T)) + dim(Im(T))$$
\end{theorem}


\begin{theorem}{}{}
    Sean $V,W$ $K$-espacios vectoriales, $T \in \mathcal{L}(V,W)$. Entonces $T$ es inyectiva si y solo si $Nuc(T) = \{\theta_V\}$.
\end{theorem}
\begin{corollary}{}{}
    Sean $V,W$ $K$-espacios vectoriales, $T \in \mathcal{L}(V,W)$. Si $V$ y $W$ son de dimensión finita y $dim_K(V) = dim_K(W)$, entonces $T$ es inyectiva si y solo si $T$ es suprayectiva.
\end{corollary}


\newpage
\section{Seccion 2}
\begin{theorem}{}{}
    Sean $V,W$ $K$-espacios vectoriales, $V$ de dimensión finita $n$. Sea $B = \{v_1, \cdots, v_n\}$ una base de $V$. Para cualesquiera $w_1, \cdots, w_n \in W$, existe una única $T \in \mathcal{L}(V,W)$ tal que $T(v_i) = w_i$ para $i = 1, \cdots, n$.
\end{theorem}
\begin{corollary}{}{}
    Sean $V,W$ $K$-espacios vectoriales con $V$ de dimensión finita $n$, $B = \{v_1, \cdots, v_n\}$ una base de $V$. Si $T, S \in \mathcal{L}(V,W)$ son tales que $T(v_i) = S(v_i)$ para $i = 1, \cdots, n$, entonces $T = S$.
\end{corollary}
\begin{definition}{}{}
    Sean $V,W$ $K$-espacios vectoriales.

    Dados $T, S \in \mathcal{L}(V,W)$, la \textbf{suma} de $T$ y $S$:
    $$T +_\mathcal{L} S: V \rightarrow W$$
    $$(T +_\mathcal{L} S)(v)= T(v) +_W S(v)$$
    Dados $T \in \mathcal{L}(V,W)$ y $\lambda \in K$, el \textbf{producto por escalares} de $T$ y $\lambda$:
    $$\lambda \cdot_\mathcal{L} T: V \rightarrow W$$
    $$(\lambda \cdot_\mathcal{L} T)(v) = \lambda \cdot_W T(v)$$
\end{definition}
\begin{proposition}{}{}
    Sean $V,W$ $K$-espacios vectoriales $T, S \in \mathcal{L}(V,W)$, $\lambda, \in K$. Entonces $T + S \in \mathcal{L}(V,W)$ y $\lambda \cdot T \in \mathcal{L}(V,W)$.
\end{proposition}
\begin{theorem}{}{}
    Sean $V,W$ $K$-espacios vectoriales. $\mathcal{L}(V,W)$ con la suma y el producto escalar definidos es un $K$-espacio vectorial.
\end{theorem}
\begin{definition}{}{}
    Sean $V,W, U$ $K$-espacios vectoriales $T\in \mathcal{L}(V,W)$, $S\in \mathcal{L}(W,U)$ Definimos $S \circ T: V \rightarrow U$ como $(S \circ T)(v) = S(T(v))$ para todo $v \in V$.

    \textbf{Nota:} La composición de funciones es asociativa
\end{definition}
\begin{theorem}{}{}
    La composición de transformaciones lineales es lineal.
\end{theorem}
\begin{obs}{}{}
    La composición de transformaciones lineales no es conmutativa.
\end{obs}
\begin{proposition}{}{}
    Sean $V,W$ $K$-espacios vectoriales, $T_1, T_2 \in \mathcal{L}(V,W)$, $S_1, S_2 \in \mathcal{L}(W,U)$. Entonces:
    \begin{enumerate}
        \item $(S_1 + S_2) \circ T_1 = S_1 \circ T_1 + S_2 \circ T_1$
        \item $S_1 \circ (T_1 + T_2) = S_1 \circ T_1 + S_1 \circ T_2$
    \end{enumerate}
\end{proposition}
\begin{obs}{}{}
    Sea $V$ un $K$-espacio vectorial. Definimos $Id_V: V \rightarrow V$ como $Id_V(v) = v$ para todo $v \in V$. $_Id_V \in \mathcal{L}(V,V)$.
\end{obs}
\begin{obs}{}{}
    $V,W$ $K$-espacios vectoriales, $T \in \mathcal{L}(V,W)$. Entonces $T \circ Id_V = T = Id_W \circ T$.
\end{obs}
\begin{definition}{}{}
    $A,B$ conjuntos, $f: A \rightarrow B$ $f$ es invertible si existe $f^{-1}: B \rightarrow A$ tal que $f^{-1} \circ f = Id_A$ y $f \circ f^{-1} = Id_B$.
    $f$ es invertible si y solo si $f$ es biyectiva.
\end{definition}
\begin{proposition}{}{}
    Sean $V, W$ $K$-espacios vectoriales, $T \in \mathcal{L}(V,W)$. Si $T$ es invertible entonces $T^{-1} \in \mathcal{L}(W,V)$.
\end{proposition}
\begin{theorem}{}{}
    Sean $V, W$ $K$-espacios vectoriales de dimensión finita con $dim_K(V) = dim_K(W)$ y $T \in \mathcal{L}(V,W)$. Las siguientes condiciones son equivalentes:
    \begin{enumerate}
        \item $T$ es invertible
        \item $T$ es inyectiva
        \item $T$ es suprayectiva
        \item Para toda $B = \{v_1, \cdots, v_n\}$ base de $V$, $\{T(v_1), T(v_2), \cdots, T(v_n)\}$ es una base de $W$.
        \item Existe una $B = \{v_1, \cdots, v_n\}$ base de $V$, tal que $\{T(v_1), T(v_2), \cdots, T(v_n)\}$ es una base de $W$.
    \end{enumerate}
\end{theorem}
\begin{theorem}{}{}
    Sean $V, W$ $K$-espacios vectoriales, $V$ de dimensión finita. Si existe $T \in \mathcal{L}(V,W)$ invertible, entonces $W$ es de dimensión finita y $dim_K(V) = dim_K(W)$.
\end{theorem}
\begin{definition}{}{}
    Sean $V,W$ $K$-espacios vectoriales. Decimos que $V$ \textbf{es isomorfo a $W$} si existe $T \in \mathcal{L}(V,W)$ invertible. En tal caso, decimos que $T$ es un \textbf{isomorfismo} de $V$ en $W$.
\begin{notation}{}{}
        $V \cong W$
\end{notation}
\end{definition}
\begin{theorem}{}{}
    Sean $V,W$ $K$-espacios vectoriales de dimensión finita. $V \cong W$ si y solo si $dim_K(V) = dim_K(W)$.
\end{theorem}
\begin{corollary}{}{}
    Si $V$ es un $K$-espacio vectorial de dimensión finita $n$, entonces $V \cong K^n$.
\end{corollary} 
\newpage



% Unidad 3 Teoria
\chapter{Transformaciones lineales y matrices}
\section{Seccion 1}
\begin{definition}{}{}
    Sea $V$ un $K$-espacio vectorial de dimensión finita $n$. Una \textbf{base ordenada de V} es una $n$-ada de vectores de $V \, \, (v_1, \cdots, v_n)$ tal que $\{v_1, \cdots, v_n\}$ es una base de $V$.
    
    \textbf{Nota: } En ocasiones $(v_1, \cdots, v_n)$ y $\{v_1, \cdots, v_n\}$ se usan indistintamente y algunos autores hacen la convención de que los subíndices indican el orden de la base.
\end{definition}
\begin{definition}{}{}
    Sea $V$ un $K$-espacio vectorial de dimensión finita $n$. Dada $B = (v_1, \cdots, v_n)$ una base ordenada de $V$, $v \in V$ el \textbf{vector de coordenadas de $v$ respecto a $B$ es}:
    \begin{equation*}
        [v]_B = \begin{pmatrix}
            \alpha_1\\
            \vdots\\
            \alpha_n
        \end{pmatrix} \in \mathcal{M}_{n\times 1}(K)
    \end{equation*}
    donde $v = \alpha_1 v_1 + \cdots + \alpha_n v_n$.
\end{definition}
\begin{obs}{}{}
    $u,v \in V$ y $\lambda \in K$. $[u + \lambda v]_B = [u]_B + \lambda [v]_B$.
\end{obs}
\begin{notation}{}{}
    Dada $A = \begin{pmatrix}
        a_{1,1} & \cdots & a_{1,n}\\
        \vdots & \ddots & \vdots\\
        a_{m,1} & \cdots & a_{m,n}
    \end{pmatrix} \in \mathcal{M}_{m \times n} (K)$, la \textbf{columna $j$-ésima de $A$} es:
    $$col_j(A) = \begin{pmatrix}
        a_{1,j}\\
        \vdots\\
        a_{m,j}
    \end{pmatrix} $$
\end{notation}
\begin{definition}{}{}
    Sean $V, W$ $K$-espacios vectoriales de dimensión finita, $B = (v_1, \cdots, v_n) \, \, \Gamma = (w_1, \cdots, w_m)$ bases ordenadas de $V$ y $W$ respectivamente, $T \in \mathcal{L}(V,W)$. La \textbf{matriz de $T$ respecto a $B$ y $\Gamma$} es una matriz $A \in \mathcal{M}_{m \times n}(K)$ tal que:
    $$col_j(A) = [T(v_j)]_\Gamma$$
    Se denotará por $[T]_{B}^\Gamma$
\end{definition}
\begin{obs}{}{}
    Si $$[T]_{B}^\Gamma = \begin{pmatrix}
        a_{1,1} & \cdots & a_{1,j} & \cdots & a_{1,n}\\
        \vdots &  &\vdots& & \vdots\\
        a_{m,1} & &a_{m,j}& & a_{m,n}
    \end{pmatrix}$$
    entonces $T(v_j) = a_{1,j}w_1 + \cdots + a_{m,j}w_m$.
\end{obs}
\begin{proposition}{}{}
    Sean $V, W$ $K$-espacios vectoriales de dimensión finita, $T \in \mathcal{L}(V,W)$, $B, \Gamma$ bases ordenadas de $V$ y $W$ respectivamente. Entonces:
    
    Para todo $v \in V$
    $$[T(v_j)]_\Gamma = [T]_B^\Gamma [v]_B$$
\end{proposition}



\newpage
\section{Seccion 2}
\begin{proposition}{}{}
    Sean $V, W$ $K$-espacios vectoriales de dimensión finita, $T, S \in \mathcal{L}(V,W)$, $B, \Gamma$ bases ordenadas de $V$ y $W$ respectivamente y $\lambda \in K$. Entonces:
    $$[\lambda S + T]_B^\Gamma = \lambda[S]_B^\Gamma + [T]_B^\Gamma$$
\end{proposition}
\begin{proposition}{}{}
    Sean $V, W$ $K$-espacios vectoriales de dimensión finita, $T, S \in \mathcal{L}(V,W)$, $B, \Gamma$ bases ordenadas de $V$ y $W$ respectivamente.

    Si $[T]_B^\Gamma = [S]_B^\Gamma$ entonces $T = S$.
\end{proposition}
\begin{proposition}{}{}
    Sean $V, W$ $K$-espacios vectoriales de dimensión finita, $T \in \mathcal{L}(V,W)$, $B, \Gamma$ bases ordenadas de $V$ y $W$ respectivamente. 
    Para toda $A \in \mathcal{M}_{m \times n}(K)$ existe $T \in \mathcal{L}(V,W)$ tal que $[T]_B^\Gamma = A$.
\end{proposition}
\begin{theorem}{}{}
    Sean $V$ y $W$ $K$-espacios vectoriales de dimensión finita, $n$ y $m$ respectivamente.
    \begin{align*}
        \mathcal{L}(V,W) &\cong \mathcal{M}_{m \times n}(K)
    \end{align*}
\end{theorem}
\begin{corollary}{}{}
    Sean $V$ y $W$ $K$-espacios vectoriales de dimensión finita, $n$ y $m$ respectivamente.
    \begin{align*}
        dim (\mathcal{L}(V, W)) &= nm
    \end{align*}
\end{corollary}
\begin{proposition}{}{}
    Sean $V, W, U$ $K$-espacios vectoriales de dimensión finita, $T \in \mathcal{L}(V,W)$, $S \in \mathcal{L}(W,U)$, $B, \Gamma, \Delta$ bases ordenadas de $V, W, U$ respectivamente. Entonces:
    $$[S \circ T]_B^\Delta = [S]_\Gamma^\Delta [T]_B^\Gamma$$
\end{proposition}
\begin{corollary}{}{}
    Sean $V, W$ $K$-espacios vectoriales de dimensión finita $dim(V) = dim(W) = n$, $T \in \mathcal{L}(V,W)$, $B, \Gamma$ bases ordenadas de $V$ y $W$ respectivamente. Entonces:

    T es invertible si y solo si $[T]_B^\Gamma$ es invertible. En este caso:
    $$[T^{-1}]_\Gamma^B = \left(\left[T\right]_B^\Gamma\right)^{-1}$$
\end{corollary}



\newpage
\section{Seccion 3}

\begin{definition}{}{}
    Sea $V$ un $K$-espacio vectorial de dimensión finita $n$, $B$ y $\Gamma$ bases ordenadas de $V$.
    La \textbf{matriz de cambio de base de B a $\Gamma$} es:
    \begin{align*}
        [id_V]_B^\Gamma \in \mathcal{M}_{n \times n}(K)\\
    \end{align*}
\end{definition}
\begin{obs}{}{}
    $[id_V]_B^\Gamma$ es invertible pues $id_V$ es una transformación lineal invertible.
    $$([id_V]_B^\Gamma)^{-1} = [id_V]_\Gamma^B$$
\end{obs}
\begin{obs}{}{}
    Para todo $v \in V$:
    \begin{align*}
        [v]_\Gamma &= [id_V(v)]_\Gamma = [id_V]_B^\Gamma [v]_B
    \end{align*}
\end{obs}
\begin{theorem}{}{}
    Sean $V, W$ $K$-espacios vectoriales de dimensiones finitas $n$ y $m$, $B$ y $B'$ bases ordenadas de $V$ y $\Gamma$ y $\Gamma'$ bases ordenadas de $W$ respectivamente. Sea $T \in \mathcal{L}(V,W)$. Entonces:
    Existen $P \in \mathcal{M}_{m \times m}(K), Q \in \mathcal{M}_{n \times n}(K)$ invertibles tales que:
    \begin{align*}
        [T]_{B'}^{\Gamma'} = P^{-1} [T]_{B}^{\Gamma} Q
    \end{align*}
\end{theorem}
\begin{corollary}{}{}
    Sea $V$ un $K$-espacio vectorial de dimensión finita $n$, $B$ y $B'$ bases ordenadas de $V$, $T \in \mathcal{L}(V, V)$.

    Existe $P \in \mathcal{M}_{n \times n}(K)$ invertible tal que:

    \begin{align*}
        [T]_{B'}^{B'} = P^{-1} [T]_{B}^{B} P
    \end{align*}
    Decimos que $[T]_{B}^{B}$ y $[T]_{B'}^{B'}$ son \textbf{matrices conjugadas}.
\end{corollary}
\begin{theorem}{}{}
    Sea $V$ un $K$-espacio vectorial de dimensión finita $n$. Si $A, C \in \mathcal{M}_{n \times n} (K)$ son tales que:
    \begin{align*}
        A &= P^{-1} C P
    \end{align*}
    Para alguna $P \in \mathcal{M}_{n \times n}(K)$ invertible, entonces existen $T \in \mathcal{L}(V,V)$ y bases ordenadas $B$ y $B'$ de $V$ tales que:
    \begin{align*}
        A &= [T]_B^B\\
        C &= [T]_{B'}^{B'}
    \end{align*}
\end{theorem}
\begin{proposition}{}{}
    Sean $V, W$ $K$-espacios vectoriales de dimensión finita, $n$ y $m$ respectivamente, $T \in \mathcal{L}(V,W)$. Existen $B$ y $\Gamma$ bases ordenadas de $V$ y $W$ respectivamente tales que:
    \begin{align*}
        [T]_B^\Gamma &= \begin{pmatrix}
            I_r & 0_{r \times (n - r)}\\
            0_{(m-r)\times r} & 0_{(m-r)\times (n-r)}
        \end{pmatrix}
    \end{align*}
    Con $r = dim(Im(T))$.
\end{proposition}
\newpage



% Unidad 4 Teoria
\chapter{Producto Interno}
\section{Sección 1}
\begin{definition}{producto-escalar}{}
    Sea $K$ un campo, $V$ un $K$-espacio vectorial. Un producto escalar en $V$ es:
    \begin{align*}
        \langle \quad , \quad \rangle : V \times V \rightarrow K 
    \end{align*}
    tal que:
    \begin{enumerate}
        \item $\langle u,v \rangle = \langle v,u \rangle \quad \forall u,v \in V$ \label{def-productoEscalar-prop-conmutatividad}
        \item $\langle u+v,w \rangle = \langle u,w \rangle + \langle v,w \rangle \quad \forall u,v,w \in V$ \label{def-productoEscalar-prop-sumas}
        \item $\langle \lambda u,v \rangle = \lambda \langle u,v \rangle \quad \forall u,v \in V, \lambda \in K$ \label{def-productoEscalar-prop-escalares}
    \end{enumerate}
\end{definition}


\begin{obs}{}{}
    Sea $w \in V$. Consideremos la función $T= \langle \quad , w\rangle$ es decir $T : V \rightarrow K$ tal que $T(v) = \langle v,w \rangle$. Entonces $T$ es lineal.
\end{obs}
\begin{proof}
    Demostraremos que $T$ es lineal, es decir, que $T$ cumple:
    \begin{align*}
        T(u+v) &= T(u) + T(v) \quad \forall u,v \in V \\
        T(\lambda u) &= \lambda T(u) \quad \forall u \in V, \lambda \in K
    \end{align*}
    Para el primer punto, sean $u,v \in V$:
    \begin{align*}
        T(u+v) &= \langle u+v, w \rangle \quad \text{por def. de la imagen de T} \\
        &= \langle u,w \rangle + \langle v,w \rangle \quad \text{por def. \ref{def-producto-escalar} de prod. escalar punto \ref{def-productoEscalar-prop-sumas}} \\
        &= T(u) + T(v) \quad \text{por def. de la imagen de T}
    \end{align*} 
    Para probar el segundo punto, sea $v \in V$ y $\lambda \in K$:
    \begin{align*}
        T(\lambda v) &= \langle \lambda v, w \rangle \quad \text{por def. de la imagen de T} \\
        &= \lambda \langle v,w \rangle \quad \text{por def. \ref{def-producto-escalar} de prod. escalar punto \ref{def-productoEscalar-prop-escalares}} \\
        &= \lambda T(v) \quad \text{por def. de la imagen de T}
    \end{align*}
\end{proof}


\begin{obs}{}{}
    También abre sumas y saca escalares en la segunda entrada.
    \begin{align*}
        \langle u, v+w \rangle &= \langle u,v \rangle + \langle u,w \rangle \quad \forall u,v,w \in V\\
        \langle u, \lambda v \rangle &= \lambda \langle u,v \rangle \quad \forall u,v \in V, \lambda \in K
    \end{align*}
\end{obs}


\begin{proof}
    Para la parte de abrir sumas en la segunda entrada:
    \begin{align*}
        \langle w, u+v \rangle &= \langle v+u, w \rangle \quad \text{por def. \ref{def-producto-escalar} de prod. escalar punto \ref{def-productoEscalar-prop-conmutatividad}} \\
        &= \langle v,w \rangle + \langle u,w \rangle \quad \text{por def. \ref{def-producto-escalar} de prod. escalar punto \ref{def-productoEscalar-prop-sumas}} 
    \end{align*}
    Para la parte de sacar escalares en la segunda entrada:
    \begin{align*}
        \langle w, \lambda u \rangle &= \langle \lambda u, w \rangle \quad \text{por \ref{def-producto-escalar} de prod. escalar punto \ref{def-productoEscalar-prop-conmutatividad}} \\
        &= \lambda \langle u,w \rangle \quad \text{por def. \ref{def-producto-escalar} de prod. escalar punto \ref{def-productoEscalar-prop-escalares}} 
    \end{align*}
\end{proof}


\begin{obs}{}{}
    $\langle \theta, v \rangle = 0 \quad \forall v \in V$ 
\end{obs}
\begin{proof}
    Sea $v \in V$:
    \begin{align*}
        \langle \theta, v \rangle &= \langle 0 \cdot \theta, v \rangle \quad \text{por prop. de espacio vectorial} \\
        & = 0 \cdot \langle \theta, v \rangle \quad \text{por def. \ref{def-producto-escalar} de prod. escalar punto \ref{def-productoEscalar-prop-escalares}} \\
        &= 0 \quad \text{por def. de $0_K$}
    \end{align*}
\end{proof}

\begin{example}{}{}
    \begin{enumerate}
        \item $K=\mathbb{R}, V= \mathbb{R} \, \, u = (x_1, \cdots, x_n), v=(y_1, \cdots, y_n) \in \mathbb{R}^n \, \, \langle u, v \rangle = x_1y_1 + \cdots + x_ny_n$ 
\begin{proof}
            Veremos que $\langle \quad, \quad \rangle$ es un producto escalar en $\mathbb{R}^n$. Procederemos a verificar la def. \ref{def-producto-escalar} de producto escalar:
            Para el punto \ref{def-productoEscalar-prop-conmutatividad}:

            Sean $u,v \in \mathbb{R}^n$:
            \begin{align*}
                \langle u,v \rangle &= x_1y_1 + \cdots + x_ny_n \quad \text{por def. de $\langle \quad, \quad \rangle$} \\
                &= y_1x_1 + \cdots + y_nx_n \quad \text{por conmutatividad de $\mathbb{R}$} \\
                &= \langle v,u \rangle \quad \text{por def. de $\langle \quad, \quad \rangle$}
            \end{align*}
\end{proof}
        \item $K=\mathbb{R}, V = C[0,1]= \{f:[0,1] \rightarrow \mathbb{R} \mid f \text{ es continua}\}$ con:
        \begin{align*}
            \langle f, g \rangle = \int_{0}^{1} f(t)g(t) dt \quad \forall f,g \in V
        \end{align*}
        \item 
    \end{enumerate}
\end{example}


\begin{definition}{}{}
    Sean $K$ un campo, $V$ un $K$-espacio vectorial, $\langle \quad, \quad \rangle$ un producto escalar en $V$. 

    Dados $u,v \in V$ decimos que $u$ es ortogonal a $v$ si $\langle u,v \rangle = 0$ y lo denotamos por $u \perp v$.

    Dado $S \subseteq V$ definimos el ortogonal a $S$ como:
    \begin{align*}
        S^{\perp} = \{ v \in V : \langle v, s \rangle = 0 \quad \forall s \in S \}
    \end{align*}
\end{definition}
\begin{obs}{}{}
    $A,B$ subconjuntos de $V$ con $A \subseteq B$ entonces $B^{\perp} \subseteq A^{\perp}$.
\end{obs}
\begin{proposition}{}{}
    Sean $K$ un campo, $V$ un $K$-espacio vectorial, $\langle , \rangle$ un producto escalar en $V$. Sea $S$ subconjunto de $V$.
    \begin{itemize}
        \item $S^{\perp}$ es un subespacio de $V$.
        \item $S^{\perp} = \langle S \rangle ^{\perp}$.
    \end{itemize}
\end{proposition}
\begin{notation}{}{}
    A $S^{\perp}$ se le llama el \textbf{subespacio ortogonal} de $S$.
\end{notation}

\begin{definition}{}{}
    Sea $K$ un campo, $V$ un $K$-espacio vectorial, $\langle , \rangle$ un producto escalar en $V$.

    Decimos que $\langle , \rangle$ es \textbf{no degenerado} si $V^{\perp} = \{ \theta_V \}$, es decir, si $v \in V$ es tal que $\langle v, w \rangle = 0 \quad \forall w \in V$ implica que $ v = \theta_V$.

    En caso contrario decimos que $\langle , \rangle$ es \textbf{degenerado}.
\end{definition}
\begin{definition}{}{}
    Sea $V$ un $\mathbb{R}$-espacio vectorial, $\langle \quad, \quad \rangle$ un producto escalar en $V$. Decimos que $\langle \quad, \quad \rangle$ es \textbf{positivo definido} si:
    \begin{enumerate}
        \item $\langle v,v \rangle \geq 0 \quad \forall v \in V$.
        \item $\langle v,v \rangle = 0$ si y solo si $v = \theta_V$.
    \end{enumerate}
\end{definition}
\begin{definition}{}{}
    Sea $K = \mathbb{R}$ o $\mathbb{C}, V$ un $K$-espacio vectorial. Una función $\langle \quad, \quad \rangle: V \times V \rightarrow K$ es un \textbf{producto interno} si:
    \begin{enumerate}
        \item $\langle u,v \rangle = \overline{\langle v,u \rangle} \quad \forall u,v \in V$.
        \item $\langle u + v, w \rangle = \langle u,w \rangle + \langle v,w \rangle \quad \forall u,v,w \in V$.
        \item $\langle \lambda u, v \rangle = \lambda \langle u,v \rangle \quad \forall u,v \in V, \lambda \in K$.
        \item $\langle v,v \rangle \geq 0 \quad \forall v \in V$ y además $\langle v,v \rangle = 0$ si y solo si $v = \theta_V$.
    \end{enumerate}
    Un espacio vectorial real o complejo con un producto interno se llama un  \textbf{espacio con producto interno}.
\end{definition}
\begin{obs}{}{}
    $\langle w, u + v \rangle = \langle w,u \rangle + \langle w,v \rangle \quad \forall u,v,w \in V$.
\end{obs}
\begin{obs}{}{}
    $\langle v, \lambda u \rangle = \overline{\lambda} \langle v,u \rangle \quad \forall u,v \in V, \lambda \in K$.
\end{obs}

\newpage
\section{Sección 2}
\begin{definition}{}{}
    Sea $V$ un $K$-espacio vectorial con producto interno, $v,w \in V, w \neq \theta_V$. El \textbf{coeficiente de fourier de $v$ respecto a $w$} es:
    \begin{align*}
        \lambda = \frac{\langle v,w \rangle}{\langle w,w \rangle}
    \end{align*}
\end{definition}
\begin{obs}{}{}
    Si $\lambda = \frac{\langle v,w \rangle}{\langle w,w \rangle}$ entonces $v - \lambda w \perp w$.
\end{obs}
\begin{definition}{}{}
    Sea $V$ un $K$-espacio vectorial con producto interno, $S$ subconjunto de $V$. Decimos que $S$ es \textbf{ortogonal} si $\langle v, w \rangle = 0 \quad \forall v,w \in S, v \neq w$.
\end{definition}
\begin{proposition}{}{}
    Sea $V$ un $K$-espacio vectorial con producto interno, $S$ subconjunto de $V$. Si $S$ es ortogonal y $\theta_V \in S$, entonces $S$ es linealmente independiente.
\end{proposition}
\begin{obs}{}{}
    Sea $\mathcal{B} = \{v_1, \cdots, v_m\}$ una base ortogonal de $V$, con $n = dim(V)$. Si $ v \in V$ se tiene que $v = \lambda_1 v_1 + \cdots + \lambda_n v_n$ con $\lambda_j$ el coeficiente de Fourier de $v$ con respecto a $v_j$.
\end{obs}
\begin{definition}{}{}
    Sea $V$ un $K$-espacio vectorial con producto interno. Dado $v \in V$ la \textbf{norma de $v$} es 
    \begin{align*}
        ||v|| = \sqrt{\langle v,v \rangle}
    \end{align*}
\end{definition}
\begin{lemma}{CAUCHY SCHWARZ}{}
    Sea $V$ un $K$-espacio vectorial con producto interno. Entonces:
    \begin{align*}
        |\langle u,v \rangle| \leq ||u|| \cdot ||v|| \quad \forall u,v \in V
    \end{align*}
\end{lemma}
\begin{proposition}{}{}
    Sea $V$ un $K$-espacio vectorial con producto interno.
    \begin{enumerate}
        \item $||v|| \geq 0 \quad \forall v \in V$ y además $||v|| = 0$ si y solo si $v = \theta_V$.
        \item $||\lambda v|| = |\lambda| \cdot ||v|| \quad \forall v \in V, \lambda \in K$.
        \item $||u+w|| \leq ||u|| + ||w|| \quad \forall u,w \in V$.
    \end{enumerate}
\end{proposition}
\begin{lemma}{Pitágoras}{}
    Sea $V$ un $K$-espacio vectorial con producto interno, $u,v \in V$ con $u \perp v$.
    \begin{enumerate}
        \item $||u+v||^2 = ||u||^2 + ||v||^2$.
        \item $||u-v||^2 = ||u||^2 + ||v||^2$.
    \end{enumerate}
\end{lemma}
\begin{definition}{}{}
    Sea $V$ un $K$-espacio vectorial con producto interno, $v \in V$. Decimos que $v$ es \textbf{unitario} si $||v|| = 1$.
\end{definition}
\begin{theorem}{Gran Schmidt}{}
    Sea $V$ un $K$-espacio vectorial con producto interno de dimensión finita. Entonces $V$ tiene una base ortogonal.
\end{theorem}

\newpage
\section{Seccion 3}

\begin{obs}{}{}{}{}
    Si $\mathcal{B} = \{v_1, \cdots, v_m, \cdots, v_n\}$ es una base y $\{v_1, \cdots, v_m\}$ es un ortogonal, entonces $\mathcal{B}' = \{v_1, \cdots, v_m, v_{m+1}', \cdots, v_n'\}$.
\end{obs}
\begin{corollary}{}{}
    Sea $V$ un $K$-espacio vectorial con producto interno de dimensión finita, $W$ subespacio de $V$, $\Gamma$ base ortogonal de $W$. Entonces existe $\mathcal{B}'$ base ortogonal de $V$ tal que $\Gamma \subseteq \mathcal{B}'$.
\end{corollary}
\begin{definition}{}{}
    Sea $V$ un $K$-espacio vectorial con producto interno. $\mathcal{B}$ subconjunto de $V$ es una \textbf{base ortonormal de $V$} si es una base ortogonal de $V$ tal que $||v|| = 1 \quad \forall v \in \mathcal{B}$.
\end{definition}
\begin{corollary}{}{}
    Sea $V$ un $K$-espacio vectorial con producto interno de dimensión finita. Entonces $V$ tiene una base ortonormal.
\end{corollary}
\begin{corollary}{}{}
    Sea $V$ un $K$-espacio vectorial con producto interno de dimensión finita, $W$ subespacio de $V$, $\Gamma$ base ortonormal de $W$. Entonces existe $\mathcal{B}'$ base ortonormal de $V$ tal que $\Gamma \subseteq \mathcal{B}'$.
\end{corollary}
\begin{definition}{}{}
    Sea $V$ un $K$-espacio vectorial con producto interno, dado $W$ un subespacio de $V$ de dimensión finita y $\Gamma = \{w_1, \cdots, w_m\}$ una base ortonormal de $W$, $v \in V$. Definimos la \textbf{proyección de $v$ en $W$ con respecto a $\Gamma$} como:
    \begin{align*}
        \pi_{W}(v) = \sum_{i=1}^{m} \langle v,w_i \rangle w_i = \langle v,w_1 \rangle w_1 + \cdots + \langle v,w_m \rangle w_m 
    \end{align*}
\end{definition}
\begin{obs}{}{}
    $v - \pi_{W}(v) \in W^\perp$.
\end{obs}
\begin{obs}{}{}
    Se verá después que $\pi_{W}(v)$ no depende de la base $\Gamma$.
\end{obs}
\begin{theorem}{}{}
    Sea $V$ un $K$-espacio vectorial con producto interno, $W$ un subespacio de $V$ de dimensión finita. Entonces:
    \begin{align*}
        V = W \oplus W^\perp
    \end{align*}
\end{theorem}
\begin{obs}{}{}
    $\pi_{W}^\Gamma(v)$ no depende de la base $\Gamma$.
\end{obs}
\begin{corollary}{}{}
    Sea $V$ un $K$-espacio vectorial con producto interno de dimensión finita, $W$ un subespacio de $V$ de dimensión finita.
    \begin{align*}
        dim(V) = dim(W) + dim(W^\perp)
    \end{align*}
    Más aún, si $\Gamma = \{w_1, \cdots, w_m\}$ es una base ortogonal de $W$ y $\mathcal{B} = \{w_1, \cdots, w_m, v_{m+1}, \cdots, v_n\}$ es una base ortogonal de $V$, entonces $\{v_{m+1}, \cdots, v_n\}$ es una base ortogonal de $W^\perp$.
\end{corollary}
\begin{theorem}{}{}
    Sea $V$ un $K$-espacio vectorial con producto interno, $W$ un subespacio de $V$ de dimensión finita. Dado $v \in V$
    \begin{align*}
        ||v - \pi_{W}(v)|| \leq ||v - w|| \quad \forall w \in W
    \end{align*}
    ($\pi_{W}(v)$ es la mejor aproximación de $v$ en $W$).
\end{theorem}
\newpage


% Unidad 1 Actividades
\chapter{Actividades}
\section{Espacios vectoriales Actividades}

\subsection{Sección 1 (Espacios vectoriales)}
\begin{exercise}{}{}
    Contesta los siguientes enunciados:
\begin{itemize}
    \item Investiga que es un campo y qué es un subcampo de un campo.
\begin{solution}{}{}
    Resuelto
\end{solution}
    \item Determina si todo subcampo de $\mathbb{C}$ contiene a $\mathbb{Q}$.
\begin{solution}{}{}
    Resuelto
\end{solution}
    \item Verifica si $\mathbb{Q}(\sqrt{2})= \{x + \sqrt{2}y \mid x,y \in \mathbb{Q}\}$ es un subcampo de $\mathbb{R}$.
\begin{solution}{}{}
    Resuelto
\end{solution}
\end{itemize}
\end{exercise}


\begin{exercise}{}{}
En $\mathbb{R}^n$ definimos las operaciones:
\begin{align*}
    u \oslash v = u -v, \quad \lambda \Diamond  v = - \lambda v, \quad \lambda \in \mathbb{R}, \quad u,v \in \mathbb{R}^n.
\end{align*}
Que axiomas de espacio vectorial se cumplen para $(\mathbb{R}^n, \oslash, \Diamond)$?
\begin{solution}{}{}
Resuelto
\end{solution}
\end{exercise}

\begin{exercise}{}{}
    Sea $V= \mathbb{R}^3$ con la suma usual. Considera ahora $K= \mathbb{Q}$ y el producto de un escalar $\lambda \in \mathbb{Q}$ por $(x,y,z) \in \mathbb{R}^3$, dado por $\lambda (x,y,z) = (\lambda x, \lambda y, \lambda z)$. ¿Es $V$ un $\mathbb{Q}$-espacio vectorial? Y si ahora se hace algo análogo con $K=\mathbb{C}$ ¿es $V$ un $C$-espacio vectorial?
\begin{solution}{}{}
Resuelto
\end{solution}
\end{exercise}

\begin{exercise}{}{}
    Considera el ejemplo $\{f  \mid f:K \rightarrow K\}$, determina si este ejemplo se puede generalizar y en vez de considerar las funciones con dominio y codominio $K$, consideramos $\{f \mid f: A \rightarrow B\}$. ¿Tiene estructura de espacio vectorial para cualesquiera conjuntos $A$ y $B$ o qué se requiere pedir a estos conjuntos para que lo sea?.
\begin{solution}{}{}
Resuelto
\end{solution}
\end{exercise}

\begin{exercise}{}{}
    Sean $K$ un campo y $V$ un $K$-espacio vectorial. Determina si dados $v \in V$, $ \lambda \in K$, el hecho de que $ \lambda v = \theta_V$ implica necesariamente que $v = \theta_V$ o $ \lambda = 0$.
\begin{solution}{}{}
Resuelto
\end{solution}
\end{exercise}




\newpage
\subsection{Sección 2 (Subespacios vectoriales)}

\begin{exercise}{}{}
    Sea $V$ un $K$-espacio vectorial y $W$ un subconjunto de $V$. Prueba o da un contraejemplo para las siguientes afirmaciones:
    \begin{enumerate}
        \item Si $W$ es cerrado bajo la suma y $\theta_V \in W$, entonces $W$ es un subespacio de $V$.
\begin{solution}{}{}
Resuelto
\end{solution}
        \item Si $W$ es cerrado bajo producto por escalar y $\theta_V \in W$, entonces $W$ es un subespacio de $V$.
\begin{solution}{}{}
Resuelto
\end{solution}
        \item Si $W$ es cerrado bajo la suma y bajo inversos aditivos, y además $\theta_V \in W$, entonces $W$ es un subespacio de $V$.        
\begin{solution}{}{}
Resuelto
\end{solution}
\end{enumerate}
\end{exercise}




\begin{exercise}{}{}
Sea $V$ un $K$-espacio vectorial y $W$ un subconjunto de $V$. Para que $W$ sea un subespacio de  $V$ ¿es necesario  pedir que $\theta_V \in W$ o se puede deducir que $W$ es cerrado bajo producto escalar?

\begin{solution}{}{}
Resuelto
\end{solution}

\end{exercise}






\begin{exercise}{}{}

Para cada uno de los siguientes incisos determina si $W \leq V$:
\begin{enumerate}
    \item $W= \{ (x,y,z) \in \mathbb{C}^3 \mid xyz=0\}$, $V= \mathbb{C}^3, \, K = \mathbb{C}$.
\begin{solution}{}{}
Resuelto (Creo)
\end{solution}

    \item $W= \{ (x,y,z) \in \mathbb{Q}^3 \mid 3x-5y+z=0\}$, $V= \mathbb{R}^3, \, K = \mathbb{R}$.
\begin{solution}{}{}
    Resuelto
\end{solution}


    \item $W= \{ f: \mathbb{R} \rightarrow \mathbb{R} \mid f(x^2) = f(x)^2\}$, $V=\{f: \mathbb{R} \rightarrow \mathbb{R}\}, \, K = \mathbb{R}$.
\begin{solution}{}{}
    Pendiente revisión
\end{solution}

    \item $W = \{A \in \mathcal{M}_{n \times n} (\mathbb{R}) \mid A^2 = A\}, \, \, V = \mathcal{M}_{n \times n} (\mathbb{R}), \, K = \mathbb{R}$.
\begin{solution}{}{}
Pendiente revisión
\end{solution}
        
\end{enumerate}

\end{exercise}


\begin{exercise}{}{}
Sea $V$ un $K$-espacio vectorial y $W$ un subconjunto de $V$. Prueba que $W \leq V$ si y solo si $W \neq \varnothing, \, \lambda u + v \in W \, \, \forall \lambda \in K, \, \, \forall u, v \in W$ 

\begin{solution}{}{}
Duda. Para probar que es no vacio podemos decir por el punto 4 de def. que existe el nuetro por lo que esta en W y por lo tanto no es vacio. ¿Esto es correcto?
\end{solution}

\end{exercise}

\begin{exercise}{}{} 
    Para cada uno de los siguientes incisos prueba que $W \leq V$:
    \begin{enumerate}

        \item $W = \{f: \mathbb{R} \rightarrow \mathbb{R} \mid f(x) = f(1-x) \forall x\}, \, \, V = \{f: \mathbb{R} \rightarrow \mathbb{R}\}, \, \, K = \mathbb{R}$.
\begin{solution}{}{}
Pendiente revisión
\end{solution}


        \item $W = \{A \in \mathcal{M}_{n \times n} (\mathbb{R}) \mid AB = BA\}, \, \, \text{ con } B\in V = \mathcal{M}_{n \times n} (\mathbb{R}), \, \, K = \mathbb{R}$.
\begin{solution}{}{}
Resuelto
\end{solution}

        \item $W = \{x = (x_i)_{i \in \mathbb{Z}^+} \mid x \text{ acotada}\}, \, \, V = \{x = (x_i)_{i \in \mathbb{Z}^+} \mid x \text{ es una sucesión en $\mathbb{R}$}\} , K = \mathbb{R}$
\begin{solution}{}{}
Duda ¿Qué es una sucesión acotada?
\end{solution}

        \item $W = \{x = (x_i)_{i \in \mathbb{Z}^+} \mid x \text{ converge}\}, \, \, V = \{x = (x_i)_{i \in \mathbb{Z}^+} \mid x \text{ es una sucesión acotada en $\mathbb{R}$}\}, \, K = \mathbb{R}$
\begin{solution}{}{}
Duda ¿Qué es una sucesión convergente?
\end{solution}

    \end{enumerate}

\end{exercise}

\begin{exercise}{}{}
Prueba que la unión de dos supespacios es subespacio si y solo si uno de ellos está contenido en el otro.

\begin{solution}{}{}
Pendiente solución. Hint tomar un vector en uno y otro que no este contenido en el otro y ver porque la union no es un espacio vectorial, así necesariamente deben estar contenidos.
\end{solution}

\end{exercise}



\begin{exercise}{}{}
Sean $V = \mathcal{M}_{3 \times 3}(\mathbb{R}), \, K = \mathbb{R}$. Considera los subespacios:
\begin{align*}
    W &= \{A \in \mathcal{M}_{3 \times 3}(\mathbb{R}) \mid A \text{es triangular superior}\}
    U &= \{A \in \mathcal{M}_{3 \times 3}(\mathbb{R}) \mid A^t = A\}
\end{align*}
Describe a $W \cap U$

\begin{solution}{}{}
Pendiente revisión.
\end{solution}

\end{exercise}


\begin{exercise}{}{}
    Sea $V = \mathcal{P}_2(\mathbb{R}), \, K = \mathbb{R}$ (el espacio vectorial de los polinomios de grado menor o igual que $2$ con coeficientes reales). Sea $S = \{1+x+x^2, -3-3x-3x^2\}$. Halla $3$ subespacios de $V$ que contengan a $S$.
\begin{solution}{}{}
Pendiente revisión. (Duda)
\end{solution}
\end{exercise}

\begin{exercise}{}{}
Considera el subconjunto de $\mathbb{R}^2$
\begin{align*}
    S = \{(2n, -5n) \mid n \in \mathbb{N}^+\}
\end{align*} 
Encuentra tres combinaciones lineales en $S$ y determina qué vectores pertenecen al conjunto de todas las combinaciones lineales de $S$.
\begin{solution}{}{}
Resuelto
\end{solution}
\end{exercise}

\newpage

\subsection{Sección 3 (Independencia lineal y generado de un conjunto)}

\begin{exercise}{}{}
    Determina si $(2,-3,7)$ pertenece al subespacio de $\mathbb{R}^3 \, \, \langle (1,0,0), (1,-1,0), (1,-1,-1) \rangle$.

\begin{solution}{}{}
Resuelto
\end{solution}
\end{exercise}


\begin{exercise}{}{}
    Determina si $\begin{pmatrix}
        1 & 2 \\
        3 & 4
    \end{pmatrix}$ pertenece al subespacio de $\mathcal{M}_{2 \times 2}(\mathbb{R})$ generado por:
    \begin{align*}
        \BigInner{\begin{pmatrix}
            1 & 1 \\
            0 & 1
        \end{pmatrix}}{\begin{pmatrix}
            1 & 0 \\
            1 & 1
        \end{pmatrix}} 
    \end{align*}

\begin{solution}{}{}
Resuelto
\end{solution}
\end{exercise}


\begin{exercise}{}{}
    Determina qué polinomios pertenecen al subespacio de $\mathcal{P}_2(\mathbb{R}) \inner{1, 1-x}{1-x+x^2}$ 
\begin{solution}{}{}
Resuelto
\end{solution}
\end{exercise}

\begin{exercise}{}{}
    Determina qué matrices pertenecen al subespacio de $\mathcal{M}_{2 \times 2}(\mathbb{R})$ generado por:
    \begin{align*}
        \BigInner{\begin{pmatrix}
            1 & 1 \\
            0 & 1
        \end{pmatrix}, \begin{pmatrix}
            1 & 0 \\
            0 & -1
        \end{pmatrix}}{\begin{pmatrix}
            0 & 0 \\
            0 & 1
        \end{pmatrix}}
    \end{align*}

\begin{solution}{}{}
Resuelto
\end{solution}
\end{exercise}



\begin{exercise}{}{}
    Determina que elementos $(x,y,z,w) \in \mathbb{R}^4$ pertenecen al subespacio de $\mathbb{R}^4$ generado por:
    \begin{align*}
        \inner{(1,1,0,0), (0,1,0,1), (0,0,1,1)}{(1,0,1,0)}
    \end{align*}

\begin{solution}{}{}
Resuelto
\end{solution}
\end{exercise}


\begin{exercise}{}{}
    Prueba que en $K^{\infty}$la lista $e_1, e_2, \cdots, e_m$ es linealmente independiente donde $m \in \mathbb{N}^+$ y $e_i$ es la sucesión que tiene un $1$ en la posición $i$ y $0$ en las demás.

\begin{solution}{}{}
Resuelto
\end{solution}
\end{exercise}

\begin{exercise}{}{}
    Sea $V$ un $K$-espacio vectorial. ¿El conjunto vacío es l.d o l.i.?
\begin{solution}{}{}
Resuelto
\end{solution}
\end{exercise}

\begin{exercise}{}{}
    Sea $V$ un $K$-espacio vectorial, sena $S'$ y $S$ con $S' \subseteq S \subseteq V$.
    \begin{enumerate}
        \item Si $S'$ o $S$ es l.d. ¿podemos saber si el otro lo es?
\begin{solution}{}{}
Resuelto
\end{solution}


        \item Si $S'$ o $S$ es l.i. ¿podemos saber si el otro lo es?
\begin{solution}{}{}
Resuelto
\end{solution}
    \end{enumerate}
\end{exercise}


\begin{exercise}{}{}
    Si un conjunto tiene al neutro ¿podemos saber si es l.d. o l.i.?

\begin{solution}{}{}
Resuelto
\end{solution}
\end{exercise}

\begin{exercise}{}{}
    Determina si en $\mathcal{P}_1(\mathbb{R})$ 
\end{exercise}



\newpage
\subsection{Sección 4}
\newpage

% Unidad 2 Actividades
\section{Transformaciones lineales. Actividades}
\subsection{Sección 1 (Transformaciones lineales)}






\newpage
\subsection{Sección 2 (Transformaciones lineales)}
\newpage


% Unidad 3 Actividades
\section{Transformaciones lineales y matrices. Actividades}
\subsection{Sección 1}


\newpage
\subsection{Sección 2}
\newpage
\subsection{Sección 3}
\newpage



% Unidad 4 Actividades
\section{Producto interno. Actividades}

\subsection{Sección 1 (Producto Interno)}
\begin{exercise}{}{}
    En los siguientes incisos determina si la función dada es producto escalar en $V$:
    
    \begin{enumerate}
        \item $K= \mathbb{R}$, $V = \mathbb{R}^2$, $\langle (a,b),(c,d) \rangle = ac-bd$ para todo $(a,b), (c,d) \in \mathbb{R}^2$.
\begin{solution}{}{}
            Para esto
\end{solution}


        \item $K= \mathbb{R}, \, \, V = \mathcal{M}_{2 \times 2} (\mathbb{R}), \, \langle A, B \rangle = tr(A + B)$ para todo $A, B \in V$.
        \item $K= \mathbb{R}, \, V= \{a+bx+cx^2 \mid a,b,c \in \mathbb{R}\}, \, \, \langle f,g \rangle = \int_{0}^{1}f'(t)g(t) \, dt$ para todos $f,g \in V$.
    \end{enumerate}
    
\end{exercise}


\newpage
\subsection{Sección 2 (Producto Interno)}


\newpage
\subsection{Sección 3 (Producto Interno)}
\newpage


\end{document}

