\chapter{Espacios vectoriales}

\section{Campos y espacios vectoriales}
\begin{definition}{Campo}{campo} 
    Sea $K$ un conjunto no vacío con dos operaciones binarias:\\ $+:K\times K\rightarrow K$ y $\cdot:K\times K\rightarrow K$. Se dice que $K$ es un campo si se cumplen las siguientes propiedades:
    \begin{enumerate}
        \item $+$ es asociativa. \label{def:campo-asociativa-aditiva}
        \item $+$ es conmutativa. \label{def:campo-conmutativa-aditiva}
        \item Existe un elemento $0_K\in K$ tal que $a+0_K=a$ para todo $a\in K$. \label{def:campo-neutro-aditivo}
        \item Para cada $a\in K$ existe un elemento $-a\in K$ tal que $a+(-a)=0$. \label{def:campo-inverso-aditivo}
        \item $\cdot$ es asociativa. \label{def:campo-asociativa-multiplicativa}
        \item $\cdot$ es conmutativa. \label{def:campo-conmutativa-multiplicativa}
        \item Existe un elemento $1\in K$ tal que $a\cdot 1=a$ para todo $a\in K$. \label{def:campo-neutro-multiplicativo}
        \item Para cada $a\in K - \{0_K\}$ existe un elemento $a^{-1}\in K$ tal que $a\cdot a^{-1}=1$. \label{def:campo-inverso-multiplicativo}
        \item $a \cdot (b+c)=a\cdot b + a\cdot c$ para todo $a,b,c\in K$. \label{def:campo-distributiva-izquierda}
    \end{enumerate}
    Llamaremos a los elementos de K escalares.
\end{definition}


\begin{example}{}{}
\begin{itemize}
    \item $\mathbb{R}$ es un campo con las operaciones usuales.
    \item $\mathbb{Q}$ es un campo con las operaciones usuales.
    \item $\mathbb{Z}$ no es un campo pues no cumple \ref{def:campo-inverso-multiplicativo}.
    \item $\mathbb{Z}_p$ con $p$ primo es un campo.
    \item $\mathbb{Q}(\sqrt{2}) = \{x + \sqrt{2}y \mid x,y \in \mathbb{Q}\}$ es un campo.
    

\end{itemize}

\end{example}

\begin{definition}{Subcampo}{subcampo}
Sean $K$ un campo, $\tilde{K} \subseteq K$. Decimos que $\tilde{K}$ es un subcampo de $K$ si $\tilde{K}$ con las operaciones restringidas de $K$ es por sí mismo un campo.
\end{definition}

\begin{example}{}{}
    \begin{itemize}
        \item $\mathbb{Q}$ es un subcampo de $\mathbb{R}$.
        \item $\mathbb{R}$ es un subcampo de $\mathbb{C}$.
    \end{itemize}
\end{example}

\begin{definition}{Espacio Vectorial}{espacio_vectorial}
    Sea $V$ un conjunto no vacío, $K$ un campo y $+:V\times V\rightarrow V$ y $\cdot:K\times V\rightarrow V$ dos operaciones. Se dice que $V$ es un espacio vectorial sobre $K $ si se cumplen las siguientes propiedades: 
    \begin{enumerate}
        \item $(u+v)+w = u+(v+w)$ para todo $u,v,w\in V$. Asociatividad.
        \item $u+v = v+u$ para todo $u,v\in V$. Conmutatividad.
        \item Existe un elemento $0_V\in V$ tal que $v+\theta_V=v$ para todo $v\in V$. Elemento neutro.
        \item Para todo $v \in V$ existe $\hat{v} \in V$ tal que $v+\hat{v}=\hat{v}+v = \theta_V$. Inverso aditivo.
        \item $1_K \cdot v = v$ para todo $v\in V$. 
        \item $\lambda \cdot(\mu \cdot v) = (\lambda \cdot \mu) \cdot v$ para todo $\lambda,\mu \in K$ y para todo $v\in V$. Distributividad.
        \item $(\lambda + \mu) \cdot v = \lambda \cdot v + \mu \cdot v$ para todo $\lambda,\mu \in K$ y para todo $v\in V$. Distributividad.
        \item $\lambda \cdot (u+v) = \lambda \cdot u + \lambda \cdot v$ para todo $\lambda \in K$ y para todo $u,v\in V$. Distributividad.
    \end{enumerate}
    Decimos que $V,+,\cdot$ es un espacio vectorial sobre el campo $K$ o un $K$-espacio vectorial. A los elementos de $V$ los llamaremos vectores.
\end{definition}


\begin{example}{}{}
    \begin{itemize}
        \item $\mathbb{R}^n$ es un $\mathbb{R}$-espacio vectorial con las operaciones usuales.
        \item $K$ campo, $K^n$ es un $K$-espacio vectorial con las operaciones usuales.
        \begin{align*}
            K^n = \{(x_1,\dots,x_n) \mid x_i \in K\} \\
            (x_1,\dots,x_n), (y_1,\dots,y_n) \in K^n \\
            (x_1,\dots,x_n) + (y_1,\dots,y_n) = (x_1+y_1,\dots,x_n+y_n) \\
            \lambda \cdot (x_1,\dots,x_n) = (\lambda x_1,\dots,\lambda x_n)
        \end{align*}

        \item $K$ campo:
        \begin{align*}
            K^{\infty} = \{(x_1, x_2, \cdots) \mid x_i \in K \, \, \forall i \in \mathbb{N}^{+}\} \\
            (x_1, x_2, \cdots), (y_1, y_2, \cdots) \in K^{\infty} \\
            (x_1, x_2, \cdots) + (y_1, y_2, \cdots) = (x_1+y_1, x_2+y_2, \cdots) \\
            \lambda \cdot (x_1, x_2, \cdots) = (\lambda x_1, \lambda x_2, \cdots)
        \end{align*} 
        Es un $K$-espacio vectorial.

        \item $K$ campo, $\mathcal{M}_{m\times n}(K)$ con $m$ renglones y $n$ columnas con las operaciones usuales de suma y producto por escalar es un $K$-espacio vectorial.

        \item $K$ campo, $K[x]$ polinomios en $x$ con coeficientes en $K$ con las operaciones usuales es un $K$-espacio vectorial.
    \end{itemize}

\end{example}

\begin{example}{}{}
    $K$ campo, $V = \{f \mid f: K \rightarrow K\}$ con las operaciones:
    \begin{align*}
        f,g \in V &\quad \lambda \in K \\
        f +_V g:& K \rightarrow K \\
        & x \mapsto f(x) +_K g(x) \\
        \lambda \cdot_V f:& K \rightarrow K \\
        & x \mapsto \lambda \cdot_K f(x)
    \end{align*}
    Es un $K$-espacio vectorial.
\end{example}

\begin{proof}
    

    \begin{enumerate}
        \item Sean $f,g,h \in V$. P.D. 
        Por construcción $(f+g)+h$, $f+(g+h)$ son funciones que van de $K \rightarrow K$. Veamos que tienen la misma regla de correspondencia.
        
        P.D. $((f+g)+h)(x) = (f+(g+h))(x) \quad \forall x \in K$.
    
        Sea $x \in K$.
        \begin{align*}
            ((f+_V g)+_V h)(x) &= (f+_V g)(x) +_K h(x) && \text{Por def. de suma en $V$} \\
            &= (f(x) +_K g(x)) +_K h(x) && \text{Por def. de suma en $V$} \\
            &= f(x) +_K (g(x) +_K h(x)) && \text{Por def. de \nameref{def:campo} punto \ref{def:campo-asociativa-aditiva} asociatividad} \\
            &= f(x) +_K (g+_V h)(x) && \text{Por def. de suma en $V$} \\
            &= (f+_V (g+_V h))(x) && \text{Por def. de suma en $V$}
        \end{align*}
        
        Por lo tanto $(f+g)+h = f+(g+h)$.


        \item Sean $f,g \in V$. P.D. $f +_V g = g +_V f$.
        Sabemos que $f +_V g: K \rightarrow K$ y $g +_V f: K \rightarrow K$. Veamos que tienen la misma regla de correspondencia.
        \begin{align*}
            (f +_V g)(x) &= f(x) +_K g(x) && \text{Por def. de suma en $V$} \\
            &= g(x) +_K f(x) && \text{Por def. de \nameref{def:campo} punto \ref{def:campo-conmutativa-aditiva} conmutatividad} \\
            &= (g +_V f)(x) && \text{Por def. de suma en $V$}
        \end{align*}
    
        Por lo tanto $f +_V g = g +_V f$.


        \item  Proponemos $\theta_V$ como elemento neutro de $V$, tal que: 
        \begin{align*}
            \theta_V: K \rightarrow K \\
            &x \mapsto 0_K
        \end{align*}
        P.D. $f +_V \theta_V = \theta_V +_V f = f \quad \forall f \in V$.
    
        Sea $f \in V$, sabemos que $f +_V \theta_V: K \rightarrow K$ y $\theta_V +_V f: K \rightarrow K$. Sea $x \in K$.
        \begin{align*}
            (f +_V \theta_V)(x) &= f(x) +_K \theta_V(x) && \text{Por def. de suma en $V$} \\
            &= f(x) +_K 0_K && \text{Por def. de $\theta_V$} \\
            &= f(x) && \text{Por def. de \nameref{def:campo} punto \ref{def:campo-neutro-aditivo} elemento neutro} \\
            &= 0_K +_K f(x) && \text{Por def. de \nameref{def:campo} punto \ref{def:campo-neutro-aditivo} elemento neutro} \\
            &= \theta_V(x) +_K f(x) && \text{Por def. de $\theta_V$} \\
            &= (\theta_V +_V f)(x) && \text{Por def. de suma en $V$}
        \end{align*}
    
        Concluimos entonces que $f +_V \theta_V = \theta_V +_V f = f$.



        \item Sea $f \in V$. Proponemos $\tilde{f}$ como inverso aditivo de $f$, tal que:
        \begin{align*}
            \tilde{f}:& K \rightarrow K \\
            &x \mapsto -f(x)
        \end{align*}
        P.D. $f +_V \tilde{f} = \tilde{f} +_V f = \theta_V$.
        Sabemos que $f +_V \tilde{f}: K \rightarrow K$ y $\tilde{f} +_V f: K \rightarrow K$. Sea $x \in K$.
    
        P.D. $(f +_V \tilde{f})(x) = \theta_V(x)$.
        \begin{align*}
            (f +_V \tilde{f})(x) &= f(x) +_K \tilde{f}(x) && \text{Por def. de suma en $V$} \\
            &= f(x) +_K (-f(x)) && \text{Por def. de $\tilde{f}$} \\
            &= 0_K && \text{Por def. de \nameref{def:campo} punto \ref{def:campo-inverso-aditivo} inverso aditivo} \\
            &= \theta_V(x) && \text{Por def. de $\theta_V$}
        \end{align*}
        Por lo tanto $f +_V \tilde{f} = \theta_V$. Analogamente se prueba que $\tilde{f} +_V f = \theta_V$.


        \item Sea $f \in V$. P.D. $1_K \cdot_V f = f$.\\
        Sabemos que $1_K \cdot_V f: K \rightarrow K$, y $f: K \rightarrow K$. Sea $x \in K$.
        \begin{align*}
            (1_K \cdot_V f)(x) &= 1_K \cdot_K f(x) && \text{Por def. de producto por escalar en $V$} \\
            &= f(x) && \text{Por def. de \nameref{def:campo} punto \ref{def:campo-neutro-multiplicativo} elemento neutro mult.}
        \end{align*}
        Por lo tanto $1_K \cdot_V f = f$.


        \item Sea $f \in V$, $\lambda, \mu \in K$. P.D. $\lambda \cdot_V (\mu \cdot_V f) = (\lambda \cdot_K \mu) \cdot_V f$.\\
        Sabemos que por construcción $\lambda \cdot_V (\mu \cdot_V f): K \rightarrow K$ y $(\lambda \cdot_K \mu) \cdot_V f: K \rightarrow K$. Sea $x \in K$.

        \begin{align*}
            (\lambda \cdot_V (\mu \cdot_V f))(x)  &= \lambda \cdot_K (\mu \cdot_V f)(x) && \text{Por def. de producto por escalar en $V$} \\
            &= \lambda \cdot_K (\mu \cdot_K f(x)) && \text{Por def. de producto por escalar en $V$} \\
            &= (\lambda \cdot_K \mu) \cdot_K f(x) && \text{Por def. de \nameref{def:campo} punto \ref{def:campo-asociativa-multiplicativa} asociatividad} \\
            &= ((\lambda \cdot_K \mu) \cdot_V f)(x) && \text{Por def. de producto por escalar en $V$}
        \end{align*}
        Por lo tanto $\lambda \cdot_V (\mu \cdot_V f) = (\lambda \cdot_K \mu) \cdot_V f$.


        \item Sea $f \in V$, $\lambda, \mu \in K$. P.D. $(\lambda +_K \mu) \cdot_V f = \lambda \cdot_V f +_V \mu \cdot_V f$.\\
        Sabemos que por construcción $(\lambda +_K \mu) \cdot_V f: K \rightarrow K$ y $\lambda \cdot_V f +_V \mu \cdot_V f: K \rightarrow K$. Sea $x \in K$.

        \begin{align*}
            ((\lambda +_K \mu) \cdot_V f)(x) &= (\lambda +_K \mu) \cdot_K f(x) && \text{Por def. de producto por escalar en $V$} \\
            &= (\lambda +_K \mu) \cdot_K f(x) && \text{Por def. de producto por escalar en $V$} \\
            &= \lambda \cdot_K f(x) +_K \mu \cdot_K f(x) && \text{Por def. de \nameref{def:campo} punto \ref{def:campo-distributiva-izquierda} distributividad} \\
            &= (\lambda \cdot_V f +_V \mu \cdot_V f)(x) && \text{Por def. de suma en $V$}
        \end{align*}

        Por lo tanto $(\lambda +_K \mu) \cdot_V f = \lambda \cdot_V f +_V \mu \cdot_V f$.


        \item Sea $f,g \in V$, $\lambda \in K$. \\
        P.D. $\lambda \cdot_V (f +_V g) = \lambda \cdot_V f +_V \lambda \cdot_V g$.\\

        Sabemos que por construcción $\lambda \cdot_V (f +_V g): K \rightarrow K$ y $\lambda \cdot_V f +_V \lambda \cdot_V g: K \rightarrow K$. Sea $x \in K$.
        \begin{align*}
            (\lambda \cdot_V (f +_V g))(x) &= \lambda \cdot_K (f +_V g)(x) && \text{Por def. de producto por escalar en $V$} \\
            &= \lambda \cdot_K (f(x) +_K g(x)) && \text{Por def. de suma en $V$} \\
            &= \lambda \cdot_K f(x) +_K \lambda \cdot_K g(x) && \text{Por def. de \nameref{def:campo} punto \ref{def:campo-distributiva-izquierda} distributividad} \\
            &= (\lambda \cdot_V f)(x) +_K (\lambda \cdot_V g)(x) && \text{Por def. de producto en $V$} \\
            &= (\lambda \cdot_V f +_V \lambda \cdot_V g)(x) && \text{Por def. de suma en $V$}
        \end{align*}

        Por lo tanto $\lambda \cdot_V (f +_V g) = \lambda \cdot_V f +_V \lambda \cdot_V g$.

    \end{enumerate}

    Al cumplir las 8 propiedades, $V$ es un $K$-espacio vectorial.
\end{proof}



\begin{proposition}{}{neutro-aditivo-unico}
    En un espacio vectorial el neutro es único.
\end{proposition}
\begin{proof}
    
    Sean $K$ un campo y $V$ un $K$-espacio vectorial. Sean $\theta_V, \theta'_V$ elementos neutros de $V$. P.D. $\theta_V = \theta'_V$.

    Note que:
    \begin{align*}
        \theta_V &= \theta_V + \theta'_V && \text{Por def. de elemento neutro} \\
        &= \theta'_V +  \theta_V && \text{Por conmutatividad en \nameref{def:campo}} \\
        &= \theta'_V && \text{Por def. de elemento neutro}
    \end{align*}
    Concluimos entonces que $\theta_V = \theta'_V$.
\end{proof}


\begin{proposition}{}{}
    En un espacio vectorial los inversos aditivos son únicos.
\end{proposition}
\begin{proof}
    Sean $K$ un campo y $V$ un $K$-espacio vectorial. Sean $v, \hat{v}, \hat{\hat{v}}$ inversos aditivos de $v$. P.D. $\hat{v} = \hat{\hat{v}}$.

    Note que:
    \begin{align*}
        \hat{v} &= \hat{v} + \theta_V && \text{Por def. de elemento neutro} \\
        &= \hat{v} + (v + \hat{\hat{v}}) && \text{Por def. de inverso aditivo} \\
        &= (\hat{v} + v) + \hat{\hat{v}} && \text{Por asociatividad en \nameref{def:campo}} \\
        &= \theta_V + \hat{\hat{v}} && \text{Por def. de inverso aditivo} \\
        &= \hat{\hat{v}} && \text{Por def. de elemento neutro}
    \end{align*}
\end{proof}


\begin{proposition}{Propiedades de cancelación}{espacio-vectorial-cancelacion-suma}
    Sean $K$ un campo y $V$ un $K$-espacio vectorial. Sean $u,v,w\in V$. Entonces:
    \begin{enumerate}
        \item Si $u+v=u+w$, entonces $v=w$.
        \item Si $u+v=w+v$, entonces $u=w$.
    \end{enumerate}
\end{proposition}
\begin{proof}
    \begin{enumerate}
        \item Supongamos que $u+v=u+w$. P.D. $v=w$.
        \begin{align*}
            v &= \theta_V + v && \text{Por def. de elemento neutro} \\
            &= (\hat{u} + u) + v && \text{Por def. de inverso aditivo} \\
            &= \hat{u} + (u + v) && \text{Por asociatividad en \nameref{def:campo}} \\
            &= \hat{u} + (u + w) && \text{Por hipótesis} \\
            &= (\hat{u} + u) + w && \text{Por asociatividad en \nameref{def:campo}} \\
            &= \theta_V + w && \text{Por def. de inverso aditivo} \\
            &= w && \text{Por def. de elemento neutro}
        \end{align*}
        \item Supongamos que $u+v=w+v$. Por la conmutatividad en \nameref{def:campo} y por la propiedad anterior y se sigue que $v=u$.
    \end{enumerate}
\end{proof}


\begin{proposition}{}{espacio-vectorial-producto-escalar-cero}
    Sean $K$ un campo y $V$ un $K$-espacio vectorial. Entonces:
    \begin{enumerate}
        \item $0_K \cdot v = \theta_V$ para todo $v\in V$.
        \item $\lambda \cdot \theta_V = \theta_V$ para todo $\lambda \in K$.
    \end{enumerate}
\end{proposition}
\begin{proof}
    \begin{enumerate}
        \item Sea $v \in V$. Entonces:
        \begin{align*}
            \theta_V + 0_K \cdot v &= 0_K \cdot v && \text{Por def. de elemento neutro} \\
            &= (0_K + 0_K) \cdot v && \text{Por def. de \nameref{def:campo} punto \ref{def:campo-neutro-aditivo} elemento neutro} \\
            &= 0_K \cdot v + 0_K \cdot v && \text{Por def. distributividad \nameref{def:campo}} \\
        \end{align*}
        Y por \nameref{prop:espacio-vectorial-cancelacion-suma} se sigue que $0_K \cdot v = \theta_V$.

        \item Sea $\lambda \in K$. Entonces:
        \begin{align*}
            \theta_V + \lambda \cdot_V \theta_V &= \lambda \cdot_V \theta_V && \text{Por def. de elemento neutro} \\
            &= \lambda \cdot_V (\theta_V + \theta_V) && \text{Por def. de elemento neutro} \\
            &= \lambda \cdot_V \theta_V + \lambda \cdot_V \theta_V && \text{Por def. distributividad \nameref{def:campo}}
        \end{align*}
        Y por \nameref{prop:espacio-vectorial-cancelacion-suma} se sigue que $\lambda \cdot_V \theta_V = \theta_V$.

    \end{enumerate}
\end{proof}




\begin{proposition}{}{espacio-vectorial-inverso-aditivo}
    Sea $K$ un campo y $V$ un $K$-espacio vectorial. 
    
    Para todo $v\in V, (-1_K)v$ es el inverso aditivo de $v$.
\end{proposition}
\begin{proof}
    Sea $v \in V$. Veamos que $(-1_K) \cdot_V v$ es su inverso aditivo.
    \begin{align*}
        v + ((-1_K) \cdot_V v) &= 1_K \cdot_V v + ((-1_K) \cdot_V v) && \text{Por def. de \nameref{def:campo} punto \ref{def:campo-neutro-multiplicativo} elemento neutro} \\
        &= (1_K + (-1_K)) \cdot_V v && \text{Por def. distributividad \nameref{def:campo}} \\
        &= 0_K \cdot_V v && \text{Por def. de \nameref{def:campo} punto \ref{def:campo-inverso-aditivo} inverso aditivo} \\
        &= \theta_V && \text{Por prop. \ref{prop:espacio-vectorial-producto-escalar-cero}}
    \end{align*}

    Por lo tanto $(-1_K) \cdot_V v$ es el inverso aditivo de $v$.
\end{proof}

\begin{notation}{}{}
    Dado $v \in V$ denotaremos por $-v$ a su inverso aditivo.
\end{notation}

\begin{corollary}{}{}
    Sean $K$ un campo y $V$ un $K$-espacio vectorial. Entonces:
    $(-\lambda)v = - (\lambda v) = \lambda (-v)$ para todo $\lambda \in K$ y para todo $v\in V$.
\end{corollary}

\begin{proof}
    Sean $\lambda \in K$ y $v \in V$. Entonces:

    \begin{align*}
        \lambda \cdot_V (-v) &= \lambda \cdot_V ((-1_K) \cdot_V v) && \text{Por prop. \ref{prop:espacio-vectorial-producto-escalar-cero} } \\
        &= (\lambda \cdot_K (-1_K)) \cdot_V v && \text{Por def. \nameref{def:espacio_vectorial} punto 6} \\
        &= (-\lambda) \cdot_V v && \text{Por def. de \nameref{def:campo} punto \ref{def:campo-inverso-multiplicativo} inverso mult.} \\
        &= ((-1_K) \cdot_K \lambda) \cdot_V v && \text{Por prop del campo} \\
        &= (-1_K) \cdot_V (\lambda \cdot_V v) && \text{Por def. \nameref{def:espacio_vectorial} punto 6} \\
        &= -1 (\lambda v) && \text{Por prop. \ref{prop:espacio-vectorial-inverso-aditivo}} \\
    \end{align*}
\end{proof}



\begin{notation}{}{}
    $K$ denotará siempre un campo.
\end{notation}

