\chapter{Espacios vectoriales}

\section{Sección 1 (Espacios vectoriales)}
\begin{definition}{Campo}{campo} 
    Sea $K$ un conjunto no vacío con dos operaciones binarias:\\ $+:K\times K\rightarrow K$ y $\cdot:K\times K\rightarrow K$. Se dice que $K$ es un campo si se cumplen las siguientes propiedades:
    \begin{enumerate}
        \item $+$ es asociativa. \label{def-campo-asociativa-aditiva}
        \item $+$ es conmutativa. \label{def-campo-conmutativa-aditiva}
        \item Existe un elemento $0_K\in K$ tal que $a+0_K=a$ para todo $a\in K$. \label{def-campo-neutro-aditivo}
        \item Para cada $a\in K$ existe un elemento $-a\in K$ tal que $a+(-a)=0$. \label{def-campo-inverso-aditivo}
        \item $\cdot$ es asociativa. \label{def-campo-asociativa-multiplicativa}
        \item $\cdot$ es conmutativa. \label{def-campo-conmutativa-multiplicativa}
        \item Existe un elemento $1\in K$ tal que $a\cdot 1=a$ para todo $a\in K$. \label{def-campo-neutro-multiplicativo}
        \item Para cada $a\in K - \{0_K\}$ existe un elemento $a^{-1}\in K$ tal que $a\cdot a^{-1}=1$. \label{def-campo-inverso-multiplicativo}
        \item $a \cdot (b+c)=a\cdot b + a\cdot c$ para todo $a,b,c\in K$. \label{def-campo-distributiva-izquierda}
    \end{enumerate}
    Llamaremos a los elementos de K escalares.
\end{definition}


\begin{example}{}{}
\begin{itemize}
    \item $\mathbb{R}$ es un campo con las operaciones usuales.
    \item $\mathbb{Q}$ es un campo con las operaciones usuales.
    \item $\mathbb{Z}$ no es un campo pues no cumple \ref{def-campo-inverso-multiplicativo}.
    \item $\mathbb{Z}_p$ con $p$ primo es un campo.
    \item $\mathbb{Q}(\sqrt{2}) = \{x + \sqrt{2}y \mid x,y \in \mathbb{Q}\}$ es un campo.
    

\end{itemize}

\end{example}

\begin{definition}{}{}\label{Subcampo}
Sean $K$ un campo, $\tilde{K} \subseteq K$ y $\tilde{K} \neq \varnothing$. Se dice que $\tilde{K}$ es un subcampo de $K$ si $\tilde{K}$ es un campo con las operaciones de $K$.
\end{definition}

\begin{proposition}{}{}
    Sea $K$ un campo, entonces:
    \begin{enumerate} \label{proposicion1_campos}
    \item $0_K$ es único. 
\begin{proof}
    Sean $0_K, 0_K' \in K$ elementos neutros aditivos.
    \begin{align*}
        a + 0_K &= a \\
        a + 0_K' &= a 
    \end{align*}
    Entonces:
    \begin{align*}
        0_K &= 0_K + 0_K' && \text{Pues $0_K'$ es neutro aditivo.} \\
        &= 0_K' && \text{Pues $0_K$ es neutro aditivo.} 
    \end{align*}
\end{proof}

    \item $-a$ es único.

\begin{proof}
    Sea $a\in K$ y sean $a_1, a_2 \in K$ inversos aditivos de $a$. Es decir:
    \begin{align*}
        a + a_1 &= a_1 + a = 0_K \\
        a + a_2 &= a_2 + a = 0_K
    \end{align*}
    Entonces:
    \begin{align*}
        a_1 &= a_1 + 0_K && \text{Pues $0_K$ es neutro aditivo.} \\
        &= a_1 + (a + a_2) && \text{Pues $a_2$ es inverso aditivo de $a$.} \\
        &= (a_1 + a) + a_2 && \text{Pues $+$ es asociativa.} \\
        &= 0_K + a_2 && \text{Pues $a_1$ es inverso aditivo de $a$.} \\
        &= a_2 && \text{Pues $0_K$ es neutro aditivo.}
    \end{align*}
    Así $a_1 = a_2$, por lo tanto el inverso aditivo es único.
\end{proof}

    \item El neutro multiplicativo $1_K$ es único.
    \item Si $x \in K-\{0_K\}$, entonces $x^{-1}$ es único.
    \end{enumerate}
\end{proposition}

\begin{definition}{}{}[Espacio Vectorial]\label{def-espacio_vectorial}
    Sea $V$ un conjunto no vacío, $K$ un campo y $+:V\times V\rightarrow V$ y $\cdot:K\times V\rightarrow V$ dos operaciones. Se dice que $V$ es un espacio vectorial sobre $K $ si se cumplen las siguientes propiedades: 
    \begin{enumerate}
        \item $(u+v)+w = u+(v+w)$ para todo $u,v,w\in V$. Asociatividad.
        \item $u+v = v+u$ para todo $u,v\in V$. Conmutatividad.
        \item Existe un elemento $0_V\in V$ tal que $v+\theta_V=v$ para todo $v\in V$. Elemento neutro.
        \item Para todo $v \in V$ existe $\hat{v} \in V$ tal que $v+\hat{v}=\hat{v}+v = \theta_V$. Inverso aditivo.
        \item $1_K \cdot v = v$ para todo $v\in V$. 
        \item $\lambda \cdot(\mu \cdot v) = (\lambda \cdot \mu) \cdot v$ para todo $\lambda,\mu \in K$ y para todo $v\in V$. Distributividad.
        \item $(\lambda + \mu) \cdot v = \lambda \cdot v + \mu \cdot v$ para todo $\lambda,\mu \in K$ y para todo $v\in V$. Distributividad.
        \item $\lambda \cdot (u+v) = \lambda \cdot u + \lambda \cdot v$ para todo $\lambda \in K$ y para todo $u,v\in V$. Distributividad.
    \end{enumerate}
    Decimos que $V,+,\cdot$ es un espacio vectorial sobre el campo $K$ o un $K$-espacio vectorial. A los elementos de $V$ los llamaremos vectores.
\end{definition}
\begin{proposition}{}{}
    En un espacio vectorial el neutro es único.
\end{proposition}
\begin{proposition}{}{}
    En un espacio vectorial los inversos aditivos son únicos.
\end{proposition}
\begin{proposition}{}{}[Propiedades de cancelación]
    Sean $K$ un campo y $V$ un $K$-espacio vectorial. Sean $u,v,w\in V$. Entonces:
    \begin{enumerate}
        \item Si $u+v=u+w$, entonces $v=w$.
        \item Si $u+v=w+v$, entonces $u=w$.
    \end{enumerate}
\end{proposition}
\begin{proposition}{}{}
    Sean $K$ un campo y $V$ un $K$-espacio vectorial. Entonces:
    \begin{enumerate}
        \item $0_K \cdot v = \theta_V$ para todo $v\in V$.
        \item $\lambda \cdot \theta_V = \theta_V$ para todo $\lambda \in K$.
    \end{enumerate}
\end{proposition}
\begin{proposition}{}{}
    Sea $K$ un campo y $V$ un $K$-espacio vectorial. 
    
    Para todo $v\in V, (-1_K)v$ es el inverso aditivo de $v$.
\end{proposition}
\begin{corollary}{}{}
    Sean $K$ un campo y $V$ un $K$-espacio vectorial. Entonces:
    $(-\lambda)v = - (\lambda v) = \lambda (-v)$ para todo $\lambda \in K$ y para todo $v\in V$.
\end{corollary}


\begin{notation}{}{}
    Dado $v \in V$ denotaremos por $-v$ a su inverso aditivo.
\end{notation}
\begin{notation}{}{}
    $K$ denotará siempre un campo.
\end{notation}

