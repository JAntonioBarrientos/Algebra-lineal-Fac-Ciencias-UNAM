
\subsection{Sección 3 (Independencia lineal y generado de un conjunto)}

\begin{exercise}{}{}
    Determina si $(2,-3,7)$ pertenece al subespacio de $\mathbb{R}^3 \, \, \langle (1,0,0), (1,-1,0), (1,-1,-1) \rangle$.

\begin{solution}{}{}
Resuelto
\end{solution}
\end{exercise}


\begin{exercise}{}{}
    Determina si $\begin{pmatrix}
        1 & 2 \\
        3 & 4
    \end{pmatrix}$ pertenece al subespacio de $\mathcal{M}_{2 \times 2}(\mathbb{R})$ generado por:
    \begin{align*}
        \BigInner{\begin{pmatrix}
            1 & 1 \\
            0 & 1
        \end{pmatrix}}{\begin{pmatrix}
            1 & 0 \\
            1 & 1
        \end{pmatrix}} 
    \end{align*}

\begin{solution}{}{}
Resuelto
\end{solution}
\end{exercise}


\begin{exercise}{}{}
    Determina qué polinomios pertenecen al subespacio de $\mathcal{P}_2(\mathbb{R}) \inner{1, 1-x}{1-x+x^2}$ 
\begin{solution}{}{}
Resuelto
\end{solution}
\end{exercise}

\begin{exercise}{}{}
    Determina qué matrices pertenecen al subespacio de $\mathcal{M}_{2 \times 2}(\mathbb{R})$ generado por:
    \begin{align*}
        \BigInner{\begin{pmatrix}
            1 & 1 \\
            0 & 1
        \end{pmatrix}, \begin{pmatrix}
            1 & 0 \\
            0 & -1
        \end{pmatrix}}{\begin{pmatrix}
            0 & 0 \\
            0 & 1
        \end{pmatrix}}
    \end{align*}

\begin{solution}{}{}
Resuelto
\end{solution}
\end{exercise}



\begin{exercise}{}{}
    Determina que elementos $(x,y,z,w) \in \mathbb{R}^4$ pertenecen al subespacio de $\mathbb{R}^4$ generado por:
    \begin{align*}
        \inner{(1,1,0,0), (0,1,0,1), (0,0,1,1)}{(1,0,1,0)}
    \end{align*}

\begin{solution}{}{}
Resuelto
\end{solution}
\end{exercise}


\begin{exercise}{}{}
    Prueba que en $K^{\infty}$la lista $e_1, e_2, \cdots, e_m$ es linealmente independiente donde $m \in \mathbb{N}^+$ y $e_i$ es la sucesión que tiene un $1$ en la posición $i$ y $0$ en las demás.

\begin{solution}{}{}
Resuelto
\end{solution}
\end{exercise}

\begin{exercise}{}{}
    Sea $V$ un $K$-espacio vectorial. ¿El conjunto vacío es l.d o l.i.?
\begin{solution}{}{}
Resuelto
\end{solution}
\end{exercise}

\begin{exercise}{}{}
    Sea $V$ un $K$-espacio vectorial, sena $S'$ y $S$ con $S' \subseteq S \subseteq V$.
    \begin{enumerate}
        \item Si $S'$ o $S$ es l.d. ¿podemos saber si el otro lo es?
\begin{solution}{}{}
Resuelto
\end{solution}


        \item Si $S'$ o $S$ es l.i. ¿podemos saber si el otro lo es?
\begin{solution}{}{}
Resuelto
\end{solution}
    \end{enumerate}
\end{exercise}


\begin{exercise}{}{}
    Si un conjunto tiene al neutro ¿podemos saber si es l.d. o l.i.?

\begin{solution}{}{}
Resuelto
\end{solution}
\end{exercise}

\begin{exercise}{}{}
    Determina si en $\mathcal{P}_1(\mathbb{R})$ 
\end{exercise}


