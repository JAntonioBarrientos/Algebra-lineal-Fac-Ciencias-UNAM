\subsection{Sección 2 (Subespacios vectoriales)}

\begin{exercise}{}{}
    Sea $V$ un $K$-espacio vectorial y $W$ un subconjunto de $V$. Prueba o da un contraejemplo para las siguientes afirmaciones:
    \begin{enumerate}
        \item Si $W$ es cerrado bajo la suma y $\theta_V \in W$, entonces $W$ es un subespacio de $V$.
\begin{solution}{}{}
Resuelto
\end{solution}
        \item Si $W$ es cerrado bajo producto por escalar y $\theta_V \in W$, entonces $W$ es un subespacio de $V$.
\begin{solution}{}{}
Resuelto
\end{solution}
        \item Si $W$ es cerrado bajo la suma y bajo inversos aditivos, y además $\theta_V \in W$, entonces $W$ es un subespacio de $V$.        
\begin{solution}{}{}
Resuelto
\end{solution}
\end{enumerate}
\end{exercise}




\begin{exercise}{}{}
Sea $V$ un $K$-espacio vectorial y $W$ un subconjunto de $V$. Para que $W$ sea un subespacio de  $V$ ¿es necesario  pedir que $\theta_V \in W$ o se puede deducir que $W$ es cerrado bajo producto escalar?

\begin{solution}{}{}
Resuelto
\end{solution}

\end{exercise}






\begin{exercise}{}{}

Para cada uno de los siguientes incisos determina si $W \leq V$:
\begin{enumerate}
    \item $W= \{ (x,y,z) \in \mathbb{C}^3 \mid xyz=0\}$, $V= \mathbb{C}^3, \, K = \mathbb{C}$.
\begin{solution}{}{}
Resuelto (Creo)
\end{solution}

    \item $W= \{ (x,y,z) \in \mathbb{Q}^3 \mid 3x-5y+z=0\}$, $V= \mathbb{R}^3, \, K = \mathbb{R}$.
\begin{solution}{}{}
    Resuelto
\end{solution}


    \item $W= \{ f: \mathbb{R} \rightarrow \mathbb{R} \mid f(x^2) = f(x)^2\}$, $V=\{f: \mathbb{R} \rightarrow \mathbb{R}\}, \, K = \mathbb{R}$.
\begin{solution}{}{}
    Pendiente revisión
\end{solution}

    \item $W = \{A \in \mathcal{M}_{n \times n} (\mathbb{R}) \mid A^2 = A\}, \, \, V = \mathcal{M}_{n \times n} (\mathbb{R}), \, K = \mathbb{R}$.
\begin{solution}{}{}
Pendiente revisión
\end{solution}
        
\end{enumerate}

\end{exercise}


\begin{exercise}{}{}
Sea $V$ un $K$-espacio vectorial y $W$ un subconjunto de $V$. Prueba que $W \leq V$ si y solo si $W \neq \varnothing, \, \lambda u + v \in W \, \, \forall \lambda \in K, \, \, \forall u, v \in W$ 

\begin{solution}{}{}
Duda. Para probar que es no vacio podemos decir por el punto 4 de def. que existe el nuetro por lo que esta en W y por lo tanto no es vacio. ¿Esto es correcto?
\end{solution}

\end{exercise}

\begin{exercise}{}{} 
    Para cada uno de los siguientes incisos prueba que $W \leq V$:
    \begin{enumerate}

        \item $W = \{f: \mathbb{R} \rightarrow \mathbb{R} \mid f(x) = f(1-x) \forall x\}, \, \, V = \{f: \mathbb{R} \rightarrow \mathbb{R}\}, \, \, K = \mathbb{R}$.
\begin{solution}{}{}
Pendiente revisión
\end{solution}


        \item $W = \{A \in \mathcal{M}_{n \times n} (\mathbb{R}) \mid AB = BA\}, \, \, \text{ con } B\in V = \mathcal{M}_{n \times n} (\mathbb{R}), \, \, K = \mathbb{R}$.
\begin{solution}{}{}
Resuelto
\end{solution}

        \item $W = \{x = (x_i)_{i \in \mathbb{Z}^+} \mid x \text{ acotada}\}, \, \, V = \{x = (x_i)_{i \in \mathbb{Z}^+} \mid x \text{ es una sucesión en $\mathbb{R}$}\} , K = \mathbb{R}$
\begin{solution}{}{}
Duda ¿Qué es una sucesión acotada?
\end{solution}

        \item $W = \{x = (x_i)_{i \in \mathbb{Z}^+} \mid x \text{ converge}\}, \, \, V = \{x = (x_i)_{i \in \mathbb{Z}^+} \mid x \text{ es una sucesión acotada en $\mathbb{R}$}\}, \, K = \mathbb{R}$
\begin{solution}{}{}
Duda ¿Qué es una sucesión convergente?
\end{solution}

    \end{enumerate}

\end{exercise}

\begin{exercise}{}{}
Prueba que la unión de dos supespacios es subespacio si y solo si uno de ellos está contenido en el otro.

\begin{solution}{}{}
Pendiente solución. Hint tomar un vector en uno y otro que no este contenido en el otro y ver porque la union no es un espacio vectorial, así necesariamente deben estar contenidos.
\end{solution}

\end{exercise}



\begin{exercise}{}{}
Sean $V = \mathcal{M}_{3 \times 3}(\mathbb{R}), \, K = \mathbb{R}$. Considera los subespacios:
\begin{align*}
    W &= \{A \in \mathcal{M}_{3 \times 3}(\mathbb{R}) \mid A \text{es triangular superior}\}
    U &= \{A \in \mathcal{M}_{3 \times 3}(\mathbb{R}) \mid A^t = A\}
\end{align*}
Describe a $W \cap U$

\begin{solution}{}{}
Pendiente revisión.
\end{solution}

\end{exercise}


\begin{exercise}{}{}
    Sea $V = \mathcal{P}_2(\mathbb{R}), \, K = \mathbb{R}$ (el espacio vectorial de los polinomios de grado menor o igual que $2$ con coeficientes reales). Sea $S = \{1+x+x^2, -3-3x-3x^2\}$. Halla $3$ subespacios de $V$ que contengan a $S$.
\begin{solution}{}{}
Pendiente revisión. (Duda)
\end{solution}
\end{exercise}

\begin{exercise}{}{}
Considera el subconjunto de $\mathbb{R}^2$
\begin{align*}
    S = \{(2n, -5n) \mid n \in \mathbb{N}^+\}
\end{align*} 
Encuentra tres combinaciones lineales en $S$ y determina qué vectores pertenecen al conjunto de todas las combinaciones lineales de $S$.
\begin{solution}{}{}
Resuelto
\end{solution}
\end{exercise}
