\chapter{Actividades}
\section{Espacios vectoriales Actividades}

\subsection{Sección 1 (Espacios vectoriales)}
\begin{exercise}{}{}
    Contesta los siguientes enunciados:
\begin{itemize}
    \item Investiga que es un campo y qué es un subcampo de un campo.
\begin{solution}{}{}
    Resuelto
\end{solution}
    \item Determina si todo subcampo de $\mathbb{C}$ contiene a $\mathbb{Q}$.
\begin{solution}{}{}
    Resuelto
\end{solution}
    \item Verifica si $\mathbb{Q}(\sqrt{2})= \{x + \sqrt{2}y \mid x,y \in \mathbb{Q}\}$ es un subcampo de $\mathbb{R}$.
\begin{solution}{}{}
    Resuelto
\end{solution}
\end{itemize}
\end{exercise}


\begin{exercise}{}{}
En $\mathbb{R}^n$ definimos las operaciones:
\begin{align*}
    u \oslash v = u -v, \quad \lambda \Diamond  v = - \lambda v, \quad \lambda \in \mathbb{R}, \quad u,v \in \mathbb{R}^n.
\end{align*}
Que axiomas de espacio vectorial se cumplen para $(\mathbb{R}^n, \oslash, \Diamond)$?
\begin{solution}{}{}
Resuelto
\end{solution}
\end{exercise}

\begin{exercise}{}{}
    Sea $V= \mathbb{R}^3$ con la suma usual. Considera ahora $K= \mathbb{Q}$ y el producto de un escalar $\lambda \in \mathbb{Q}$ por $(x,y,z) \in \mathbb{R}^3$, dado por $\lambda (x,y,z) = (\lambda x, \lambda y, \lambda z)$. ¿Es $V$ un $\mathbb{Q}$-espacio vectorial? Y si ahora se hace algo análogo con $K=\mathbb{C}$ ¿es $V$ un $C$-espacio vectorial?
\begin{solution}{}{}
Resuelto
\end{solution}
\end{exercise}

\begin{exercise}{}{}
    Considera el ejemplo $\{f  \mid f:K \rightarrow K\}$, determina si este ejemplo se puede generalizar y en vez de considerar las funciones con dominio y codominio $K$, consideramos $\{f \mid f: A \rightarrow B\}$. ¿Tiene estructura de espacio vectorial para cualesquiera conjuntos $A$ y $B$ o qué se requiere pedir a estos conjuntos para que lo sea?.
\begin{solution}{}{}
Resuelto
\end{solution}
\end{exercise}

\begin{exercise}{}{}
    Sean $K$ un campo y $V$ un $K$-espacio vectorial. Determina si dados $v \in V$, $ \lambda \in K$, el hecho de que $ \lambda v = \theta_V$ implica necesariamente que $v = \theta_V$ o $ \lambda = 0$.
\begin{solution}{}{}
Resuelto
\end{solution}
\end{exercise}



