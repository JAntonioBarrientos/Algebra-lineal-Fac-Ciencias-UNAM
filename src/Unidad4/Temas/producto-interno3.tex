\section{Seccion 3}

\begin{obs}{}{}{}{}
    Si $\mathcal{B} = \{v_1, \cdots, v_m, \cdots, v_n\}$ es una base y $\{v_1, \cdots, v_m\}$ es un ortogonal, entonces $\mathcal{B}' = \{v_1, \cdots, v_m, v_{m+1}', \cdots, v_n'\}$.
\end{obs}
\begin{corollary}{}{}
    Sea $V$ un $K$-espacio vectorial con producto interno de dimensión finita, $W$ subespacio de $V$, $\Gamma$ base ortogonal de $W$. Entonces existe $\mathcal{B}'$ base ortogonal de $V$ tal que $\Gamma \subseteq \mathcal{B}'$.
\end{corollary}
\begin{definition}{}{}
    Sea $V$ un $K$-espacio vectorial con producto interno. $\mathcal{B}$ subconjunto de $V$ es una \textbf{base ortonormal de $V$} si es una base ortogonal de $V$ tal que $||v|| = 1 \quad \forall v \in \mathcal{B}$.
\end{definition}
\begin{corollary}{}{}
    Sea $V$ un $K$-espacio vectorial con producto interno de dimensión finita. Entonces $V$ tiene una base ortonormal.
\end{corollary}
\begin{corollary}{}{}
    Sea $V$ un $K$-espacio vectorial con producto interno de dimensión finita, $W$ subespacio de $V$, $\Gamma$ base ortonormal de $W$. Entonces existe $\mathcal{B}'$ base ortonormal de $V$ tal que $\Gamma \subseteq \mathcal{B}'$.
\end{corollary}
\begin{definition}{}{}
    Sea $V$ un $K$-espacio vectorial con producto interno, dado $W$ un subespacio de $V$ de dimensión finita y $\Gamma = \{w_1, \cdots, w_m\}$ una base ortonormal de $W$, $v \in V$. Definimos la \textbf{proyección de $v$ en $W$ con respecto a $\Gamma$} como:
    \begin{align*}
        \pi_{W}(v) = \sum_{i=1}^{m} \langle v,w_i \rangle w_i = \langle v,w_1 \rangle w_1 + \cdots + \langle v,w_m \rangle w_m 
    \end{align*}
\end{definition}
\begin{obs}{}{}
    $v - \pi_{W}(v) \in W^\perp$.
\end{obs}
\begin{obs}{}{}
    Se verá después que $\pi_{W}(v)$ no depende de la base $\Gamma$.
\end{obs}
\begin{theorem}{}{}
    Sea $V$ un $K$-espacio vectorial con producto interno, $W$ un subespacio de $V$ de dimensión finita. Entonces:
    \begin{align*}
        V = W \oplus W^\perp
    \end{align*}
\end{theorem}
\begin{obs}{}{}
    $\pi_{W}^\Gamma(v)$ no depende de la base $\Gamma$.
\end{obs}
\begin{corollary}{}{}
    Sea $V$ un $K$-espacio vectorial con producto interno de dimensión finita, $W$ un subespacio de $V$ de dimensión finita.
    \begin{align*}
        dim(V) = dim(W) + dim(W^\perp)
    \end{align*}
    Más aún, si $\Gamma = \{w_1, \cdots, w_m\}$ es una base ortogonal de $W$ y $\mathcal{B} = \{w_1, \cdots, w_m, v_{m+1}, \cdots, v_n\}$ es una base ortogonal de $V$, entonces $\{v_{m+1}, \cdots, v_n\}$ es una base ortogonal de $W^\perp$.
\end{corollary}
\begin{theorem}{}{}
    Sea $V$ un $K$-espacio vectorial con producto interno, $W$ un subespacio de $V$ de dimensión finita. Dado $v \in V$
    \begin{align*}
        ||v - \pi_{W}(v)|| \leq ||v - w|| \quad \forall w \in W
    \end{align*}
    ($\pi_{W}(v)$ es la mejor aproximación de $v$ en $W$).
\end{theorem}