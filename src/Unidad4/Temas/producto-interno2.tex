\section{Sección 2}
\begin{definition}{}{}
    Sea $V$ un $K$-espacio vectorial con producto interno, $v,w \in V, w \neq \theta_V$. El \textbf{coeficiente de fourier de $v$ respecto a $w$} es:
    \begin{align*}
        \lambda = \frac{\langle v,w \rangle}{\langle w,w \rangle}
    \end{align*}
\end{definition}
\begin{obs}{}{}
    Si $\lambda = \frac{\langle v,w \rangle}{\langle w,w \rangle}$ entonces $v - \lambda w \perp w$.
\end{obs}
\begin{definition}{}{}
    Sea $V$ un $K$-espacio vectorial con producto interno, $S$ subconjunto de $V$. Decimos que $S$ es \textbf{ortogonal} si $\langle v, w \rangle = 0 \quad \forall v,w \in S, v \neq w$.
\end{definition}
\begin{proposition}{}{}
    Sea $V$ un $K$-espacio vectorial con producto interno, $S$ subconjunto de $V$. Si $S$ es ortogonal y $\theta_V \in S$, entonces $S$ es linealmente independiente.
\end{proposition}
\begin{obs}{}{}
    Sea $\mathcal{B} = \{v_1, \cdots, v_m\}$ una base ortogonal de $V$, con $n = dim(V)$. Si $ v \in V$ se tiene que $v = \lambda_1 v_1 + \cdots + \lambda_n v_n$ con $\lambda_j$ el coeficiente de Fourier de $v$ con respecto a $v_j$.
\end{obs}
\begin{definition}{}{}
    Sea $V$ un $K$-espacio vectorial con producto interno. Dado $v \in V$ la \textbf{norma de $v$} es 
    \begin{align*}
        ||v|| = \sqrt{\langle v,v \rangle}
    \end{align*}
\end{definition}
\begin{lemma}{CAUCHY SCHWARZ}{}
    Sea $V$ un $K$-espacio vectorial con producto interno. Entonces:
    \begin{align*}
        |\langle u,v \rangle| \leq ||u|| \cdot ||v|| \quad \forall u,v \in V
    \end{align*}
\end{lemma}
\begin{proposition}{}{}
    Sea $V$ un $K$-espacio vectorial con producto interno.
    \begin{enumerate}
        \item $||v|| \geq 0 \quad \forall v \in V$ y además $||v|| = 0$ si y solo si $v = \theta_V$.
        \item $||\lambda v|| = |\lambda| \cdot ||v|| \quad \forall v \in V, \lambda \in K$.
        \item $||u+w|| \leq ||u|| + ||w|| \quad \forall u,w \in V$.
    \end{enumerate}
\end{proposition}
\begin{lemma}{Pitágoras}{}
    Sea $V$ un $K$-espacio vectorial con producto interno, $u,v \in V$ con $u \perp v$.
    \begin{enumerate}
        \item $||u+v||^2 = ||u||^2 + ||v||^2$.
        \item $||u-v||^2 = ||u||^2 + ||v||^2$.
    \end{enumerate}
\end{lemma}
\begin{definition}{}{}
    Sea $V$ un $K$-espacio vectorial con producto interno, $v \in V$. Decimos que $v$ es \textbf{unitario} si $||v|| = 1$.
\end{definition}
\begin{theorem}{Gran Schmidt}{}
    Sea $V$ un $K$-espacio vectorial con producto interno de dimensión finita. Entonces $V$ tiene una base ortogonal.
\end{theorem}
